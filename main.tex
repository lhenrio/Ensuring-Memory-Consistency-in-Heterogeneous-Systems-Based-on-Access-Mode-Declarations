% !TeX spellcheck = en_GB
%% 
%% Copyright 2007-2018 Elsevier Ltd
%% 
%% This file is part of the 'Elsarticle Bundle'.
%% ---------------------------------------------
%% 
%% It may be distributed under the conditions of the LaTeX Project Public
%% License, either version 1.2 of this license or (at your option) any
%% later version.  The latest version of this license is in
%%    http://www.latex-project.org/lppl.txt
%% and version 1.2 or later is part of all distributions of LaTeX
%% version 1999/12/01 or later.
%% 
%% The list of all files belonging to the 'Elsarticle Bundle' is
%% given in the file `manifest.txt'.
%% 

%% Template article for Elsevier's document class `elsarticle'
%% with numbered style bibliographic references
%% SP 2008/03/01
%%
%% 
%%
%% $Id: elsarticle-template-num.tex 64 2013-05-15 12:23:51Z rishi $
%%
%%
\documentclass[preprint,12pt]{elsarticle}

%% Use the option review to obtain double line spacing
%% \documentclass[authoryear,preprint,review,12pt]{elsarticle}

%% Use the options 1p,twocolumn; 3p; 3p,twocolumn; 5p; or 5p,twocolumn
%% for a journal layout:
%% \documentclass[final,1p,times]{elsarticle}
%% \documentclass[final,1p,times,twocolumn]{elsarticle}
%% \documentclass[final,3p,times]{elsarticle}
%% \documentclass[final,3p,times,twocolumn]{elsarticle}
%% \documentclass[final,5p,times]{elsarticle}
%% \documentclass[final,5p,times,twocolumn]{elsarticle}

%% For including figures, graphicx.sty has been loaded in
%% elsarticle.cls. If you prefer to use the old commands
%% please give \usepackage{epsfig}

%% The amssymb package provides various useful mathematical symbols

\usepackage{amsmath,amssymb,amsfonts}
\usepackage{algorithmic}
\usepackage{graphicx}
\usepackage{textcomp}
\usepackage{xcolor}
\usepackage[utf8]{inputenc}
\usepackage{mathpartir,color}
\usepackage{stmaryrd}
\usepackage{amsthm}
\usepackage{graphicx}
\newcommand{\TODO}[1]{\textcolor{red}{\textbf{[TODO:#1]}}}
\newcommand{\symb}[1]{\textit{#1}} 
\newcommand{\noop}{\symb{Noop}}
\newcommand{\Push}{\symb{Push}}
\newcommand{\Pull}{\symb{Pull}}
\newcommand{\while}{\symb{While}}
\newcommand{\cond}{\symb{cond}}
\DeclareMathOperator{\vars}{vars}
\newcommand{\isvalid}{\symb{isValid}}
\newcommand{\isremvalid}{\symb{remIsValid}}
\newcommand{\rem}[1]{\symb{rem}(#1)}
\newcommand{\IF}[3]{\symb{if}\,(#1)~#2~\symb{else}~#3 }
\newcommand{\feval}[2]{\llbracket#1\rrbracket_{#2}}
\newcommand{\True}{{\tt True}}						
\newcommand{\False}{{\tt False}}				
\newcommand{\transl}[1]{\llbracket#1\rrbracket}
\newtheorem{definition}{Definition}
\newtheorem{Property}{Property}
\newtheorem{Theorem}{Theorem}
\newcommand{\abs}[1]{#1^\#}
\newcommand{\AM}{\mathcal{M}}
\newcommand{\Prog}{\mathcal{P}}
\usepackage{etoolbox}
\DeclareMathOperator{\range}{range}
\DeclareMathOperator{\dom}{dom}
\newcommand{\Overlap}[1]{O(#1)}

%% The amsthm package provides extended theorem environments
%% \usepackage{amsthm}

%% The lineno packages adds line numbers. Start line numbering with
%% \begin{linenumbers}, end it with \end{linenumbers}. Or switch it on
%% for the whole article with \linenumbers.
\usepackage{lineno}

\journal{Journal of Logical and Algebraic Methods in Programming}

\begin{document}

\begin{frontmatter}

%% Title, authors and addresses

%% use the tnoteref command within \title for footnotes;
%% use the tnotetext command for theassociated footnote;
%% use the fnref command within \author or \address for footnotes;
%% use the fntext command for theassociated footnote;
%% use the corref command within \author for corresponding author footnotes;
%% use the cortext command for theassociated footnote;
%% use the ead command for the email address,
%% and the form \ead[url] for the home page:
%% \title{Title\tnoteref{label1}}
%% \tnotetext[label1]{}
%% \author{Name\corref{cor1}\fnref{label2}}
%% \ead{email address}
%% \ead[url]{home page}
%% \fntext[label2]{}
%% \cortext[cor1]{}
%% \address{Address\fnref{label3}}
%% \fntext[label3]{}
\title{Leveraging Access Mode 
Declarations in a Model for Memory Consistency in Heterogeneous Systems }
%\title{Ensuring Memory Consistency in Heterogeneous Systems Based on Access Mode
%Declarations }


%% use optional labels to link authors explicitly to addresses:
%% \author[label1,label2]{}
%% \address[label1]{}
%% \address[label2]{}
\author[i3s]{Ludovic Henrio\corref{cor1}}
\ead{ludovic.henrio@cnrs.fr}
\author[liu]{Christoph Kessler}
\ead{christoph.kessler@liu.se}
\author[liu]{Lu Li}
\ead{lu.li@liu.se}
\cortext[cor1]{Corresponding author}

\address[i3s]{Universit\'e~C\^ote~d'Azur, CNRS, I3S, France.}
\address[liu]{University of Linköping, Sweden}

\begin{abstract}
Running a program on disjoint memory spaces requires to address memory consistency 
issues and to perform  transfers so that the program always accesses  the right 
data. Several approaches exist to ensure the consistency of the memory accessed,  we 
are interested here in the verification of a declarative approach where each component of 
a computation is annotated with an access mode declaring which part of the memory is 
read or written by the component. The programming framework uses the component
annotations to guarantee the validity of the memory accesses. This is the mechanism used in VectorPU, a C++ library for programming CPU-GPU heterogeneous  systems 
and this article proves the correctness of the software cache-coherence mechanism used in 
the library.  The formalism we propose also takes into account arrays for which a single validity status is stored for the whole array; additional mechanisms for dealing with overlapping arrays are also studied. Beyond the scope of VectorPU, this article can be considered as a simple 
and effective formalisation of memory consistency mechanisms based on the explicit 
declaration of the effect of each component on each memory space.

\end{abstract}

\begin{keyword}
%% keywords here, in the form: keyword \sep keyword
Memory consistency \sep CPU-GPU heterogeneous systems \sep  
data transfer \sep  software caching \sep  cache coherence 
%% PACS codes here, in the form: \PACS code \sep code

%% MSC codes here, in the form: \MSC code \sep code
%% or \MSC[2008] code \sep code (2000 is the default)

\end{keyword}

\end{frontmatter}

\section{Introduction}

% !TeX spellcheck = en_GB
Heterogeneous computer systems, such as traditional CPU-GPU based systems, 
% keep it simple and remove:
% require different programming models for
% different execution unit types, and they often 
often expose disjoint memory spaces to the programmer,
such as main memory and device memory, with the need
to explicitly transfer data between these.
% \TODO{I was confused with the next 3 sentecnes , please check this version}
The different memories usually 
require different memory access operations and
different pointer types. % on CPU and accelerator side.
% Done, removed the rest as it might confuse:   even in programming
% environments where a single-source program
% involves both CPU and accelerator code.
Also, encoding memory transfers as message passing communications
leads to low-level code that is more error-prone. 
% While the former problem is often solved by 
% compiler support (as in OpenACC, OpenMP4.5 and
% domain-specific languages)
% or other high-level programming
% abstractions (e.g.\ skeleton programming as in 
% SkePU or SkelCL), e.g. 
A commonly used software technique to abstract away the
distributed memory, the explicit message passing,
and the asymmetric memory access mechanisms
consists in providing the programmer with an 
object-based shared memory emulation. For CPU-GPU systems,
this can be done in the form of special data-containers,
which are generic, STL-like data abstractions such as 
\verb+vector<...>+ that 
% \TODO{I am a bit confused with the ``wrap aggregate''} 
% is multi-element better?
wrap multi-element data structures such as arrays. These 
data-container objects
 internally perform transparent, coherent
software caching
of (subsets of) accessed elements in the different memories 
so they can be reused (as long as not invalidated) 
in order to
avoid unnecessary data transfers. Such data-containers 
(sometimes also referred to as ''smart'' containers as they can
transparently perform data transfer and memory allocation optimizations
\cite{Dastgeer-IJPP15}) 
are provided in a number of programming frameworks 
for heterogeneous systems, such as
StarPU \cite{StarPU} and SkePU~\cite{Enmyren10,Dastgeer-IJPP15}. StarPU is a 
C-based library that provides API functions to
define multi-variant tasks for dynamic scheduling
where the data containers are used for modeling 
the operand data-flow
among the dynamically scheduled tasks. 
SkePU defines device-independent 
multi-backend skeletons like map, reduce,
scan, stencil etc.\ where operands are
passed to skeleton calls within data containers.

VectorPU \cite{VectorPU-2017} is a recent C++-only
open-source
programming framework for CPU-GPU heterogeneous systems. 
VectorPU relies on the specification of 
\textit{components}, which are functions that contain kernels for execution
on either CPU or GPU. Programming in VectorPU is thus not restricted 
to using predefined skeletons like SkePU, 
but leads to more high-level and more concise code than StarPU. 
Like StarPU, VectorPU requires the programmer
to annotate each operand of a component
%\TODO{passed to components? (I find the ``operand'' a bit abrupt here)} 
with the access mode (read, write, or both) including the 
accessing unit (CPU, GPU), and uses smart data containers for automatic transparent
software caching based on this access mode information.

The implementation of VectorPU makes excessive use of static metaprogramming; this provides a light-weight realization of the access mode annotations and of the software caching, 
which only require a standard C++ compiler. Emulating these 
light-weight
component and access mode constructs without additional language
and compiler support (in contrast to, e.g., OpenACC or OpenMP), 
leads however to some compromises concerning the possibility to perform static analysis.
In particular, VectorPU has no explicit type system for the
access modes, as these are not known to the C++ compiler.

In this paper, we  formalize access modes
and data transfers in CPU-GPU heterogeneous systems and prove 
the correctness of the software
cache coherence mechanism used in VectorPU.
The contributions of this paper are:

\begin{itemize}
\item A simple effect system modeling the semantics of memory
   accesses and communication in a CPU-GPU heterogeneous system,
\item A small calculus expressing different memory
   accesses and their composition across program traces. 
\item The interpretation of VectorPU operations as higher-level statements
    that can be translated into the core calculus,
\item A proof that, if all memory accesses are performed 
    through VectorPU operations, the memory cannot reach an 
    inconsistent state and all memory accesses succeed,
\item The abstraction necessary to take into account arrays, possibly overlapping, in the formalism.
\end{itemize}

This article is an extended version of \cite{HKL-4PAD2018}, with two main additions. First the relationship between the formal results and the VectorPU implementation is  detailed, illustrating the impact of the proven results on the behaviour of the library. Second, the theoretical framework is extended to take into account the fact that manipulated arrays may overlap and that the consistency mechanism must take this information into account to be correct. While overlapping arrays are not yet supported by VectorPU, based on the formal model we develop, we show how a simple extension of the library could provide support for overlapping arrays.

This paper is organized as follows:
Section~\ref{VectorPU} reviews VectorPU as far as required for  this paper, for further information we refer to \cite{VectorPU-2017}.
Section~\ref{sec:Formal} provides our formalization of VectorPU programs and
their semantics, and proves that the coherence mechanism used in
VectorPU is sound. Section~\ref{sec:RW} discusses related work, and 
Section~\ref{sec:conclusion} concludes.

\begin{figure}
\begin{center}
\includegraphics[width=0.64\textwidth]{img/CPU-GPU.png}
\caption{\label{fig:CPU-GPU}A GPU-based system with distributed address space}
\end{center}
\end{figure}

\section{VectorPU}\label{VectorPU}

In heterogeneous systems with separate address spaces, for example in 
many GPU-based systems, a general-purpose processor
(CPU) with direct access to main memory is connected by some network
(e.g., PCIe bus) to one or several accelerators
(e.g., GPUs) each having its own device memory,
see Figure~\ref{fig:CPU-GPU}. Native programming models
for such systems such as CUDA typically expose the distributed address spaces to the programmer, who has to write explicit 
code for data transfers and device memory management.
Often, programs for such systems  must be organized in multiple source files
as different programming models and different toolchains
are to be used for different types of execution unit.
This enforces a low-level programming style. 
Accordingly, a number of single-source 
programming approaches have been proposed that abstract away the distribution 
by providing a virtual shared address space. Examples include directive-based
language extensions such as OpenACC and OpenMP4.5, and C++-only approaches 
such as the library-based skeleton programming framework SkePU \cite{Dastgeer-IJPP15}
and the recent macro-based framework \textit{VectorPU}.

\textit{VectorPU} \cite{VectorPU-2017} is an open-source\footnote{http://www.ida.liu.se/labs/pelab/vectorpu, https://github.com/lilu09/vectorpu} lightweight C++-only 
high-level programming layer
for writing single-source heterogeneous programs for Nvidia CUDA GPU-based systems.
Aggregate operand data structures 
passed into or out of  function calls are
to be wrapped by special data containers known to VectorPU.
VectorPU currently provides one generic data container,
called \verb+vector<...>+,
with multiple variants that 
eliminate the overhead of managing heterogeneity and distribution when not required (e.g., when no GPU is available). 
\verb+vector<...>+ inherits functionality from STL \verb.vector. 
and from Nvidia Thrust \verb.vector., and
wraps a C++ array allocated in main memory. 
VectorPU automatically creates on demand
copies of to-be accessed elements in device memory and keeps all copies coherent using
a simple coherence protocol, data transfers are only performed when needed. 

VectorPU programs are organized as a set of C++ functions, some of which
might internally use device-specific programming CUDA constructs\footnote{%
VectorPU allows to directly annotate a CUDA kernel function, in addition to annotating its C++ wrapper function.} while others
are expected to execute on the host, using one or possibly multiple cores.
VectorPU \emph{components} are functions that are supposed to contain (CPU or device)
kernel functionality and for which  operands are passed as VectorPU data container objects. 
Components and the types of execution units that
access their operands are declared 
by annotating the operands of the function, either at a call of the function
or for the formal parameters in the function's declaration, 
with VectorPU \emph{access mode specifiers}. For example, in contrast,  SkePU \cite{Enmyren10}  overloads element
access and iterator operations so that monitored 
accesses are also possible on demand in non-componentized (i.e., 
ordinary C++) CPU code.
VectorPU only relies on access mode annotations 
to perform lazy data transfer,
not knowing when data is going to be accessed inside a component.
% however, SkePU knows by overloading element accesses.
% By overloading element
% access and iterator operations, 

Table~\ref{tab:modes} summarizes the access mode annotations
currently defined for VectorPU. The access mode specifiers,
such as \texttt{R} (read on CPU), \texttt{W} (write on CPU), \texttt{RW} 
(update, i.e., both read and write, on CPU), \texttt{GR} (read on GPU) and so forth,
are available both as annotations of function signatures and
as C++ preprocessor macros that expand at compilation into (possibly, device-specific) C++ pointer
expressions and side effects that allow to generate device specific access code
and use device-specific pointer types for the chosen execution unit. 
For instance, \texttt{GW(x)} expands to a GPU pointer to
the GPU device copy of \texttt{x},
which might be dereferenced for GPU writing accesses to \texttt{x},
such as the GPU code: \verb:*( GW(x) + 2 ) = 3.14:.
\texttt{GWI(x)} evaluates to a Thrust-compatible iterator onto the 
GPU device copy of \texttt{x}, and \texttt{WEI(x)} to an iterator-end reference
to the last element of \texttt{x} on CPU side. The current VectorPU prototype implementation does not
(yet) check access-mode annotations in signatures of externally defined functions.
%
It is also possible to specify partial access of a \verb:vector:
% that expand into pointers to the first and last element
% of an interval of elements to be accessed, 
instead of the
entire \verb.vector. data structure. The current
VectorPU implementation does not (yet) support coherence for 
\emph{overlapping}
intervals of elements resulting from multiple (partial) accesses
some of which (may) access the same element.
A solution for this problem has been described for SkePU
smart containers by Dastgeer~\cite{Dastgeer-IJPP15}. Section~\ref{sec:overlap-array} details a solution for handling overlapping arrays in VectorPU.
%


\begin{table}
\caption{\label{tab:modes}VectorPU access mode annotations for a parameter  \cite{VectorPU-2017}}

\begin{center}
\begin{tabular}{|lll|}
\hline
Access Mode & On Host & On Device \\
\hline
Read pointer & \texttt{R} & \texttt{GR} \\
Write pointer & \texttt{W} & \texttt{GW} \\
Read and Write pointer & \texttt{RW} & \texttt{GRW} \\
Read Iterator & \texttt{RI} & \texttt{GRI} \\
Read End Iterator & \texttt{REI} & \texttt{GREI}\\
Write Iterator & \texttt{WI} & \texttt{GWI}\\
Write End Iterator & \texttt{WEI} & \texttt{GWEI}\\
Read and Write Iterator & \texttt{RWI} & \texttt{GRWI}\\
Read and Write End Iterator & \texttt{RWEI} & \texttt{GRWEI}\\
Not Applicable & \texttt{NA} & \texttt{NA} \\
\hline
\end{tabular}
\vspace{-3ex}
\end{center}
\end{table}


The following example (adapted from \cite{VectorPU-2017}) 
of a CUDA kernel wrapped in an 
 annotated function \verb.bar. shows the use of 
VectorPU access mode annotations at function declaration:

{\footnotesize \begin{verbatim}
// Example (annotations at function declaration): 
__global__
void bar ( const float *x [[GR]], float *y [[GW]],
                 float *z [[GRW]], int size )
{ ... CUDA kernel code ... }
\end{verbatim}}

Here, the operand array pointed to by \verb.x. may be read (only) by the GPU within \verb.bar.,
operand array \verb.y. must be written (only) by the GPU, and 
operand array \verb.z. may be read and/or written by the GPU.
When calling \verb.bar., the
first three operands are  passed as VectorPU \verb.vector. 
container objects.
The \verb.size. formal parameter is a scalar (not a data container), so it
will be available on GPU on a copy-in basis but no coherence will
be provided for it by VectorPU.

It is also possible to put the annotations into a call, and hence characterize a function as a VectorPU component:


{\footnotesize \begin{verbatim}
// declare a CPU function:
void foo ( const float *x, float *y, float *z, int size );

// declare three vectors:
vectorpu::vector<float> vx(100), vy(100), vz(100);

// call to VectorPU annotated function foo:
foo ( R( vx ), W( vy ), RW( vz ), size ) ;
\end{verbatim}}

Here, the access mode specifiers and the resulting coherence policy
 only apply to that particular invocation of \verb-foo-, while other
invocations of \verb-foo- might use different access mode specifiers.

\begin{example}
The following example shows how to use iterators:

{\footnotesize \begin{verbatim}
vectorpu::vector<My_Type> vx(N);
std::generate( WI(vx), WEI(vx), RandomNumber );
thrust::sort( GRWI(vx), GRWEI(vx));
std::copy( RI(vx), REI(vx), ostream_iterator<My_Type>(cout, ""));
\end{verbatim}}

\noindent 
where \verb.std::generate. is a CPU function filling a section between
two addresses with values (here, random numbers),
and \verb.thrust::sort. denotes the GPU sorting functionality
provided by the Nvidia Thrust library. 
\end{example}
% CK 190926: moved to new subsection 2.1 on pvectors
% % New 180916 CK:
% Using iterators, it is possible to define, in VectorPU, references to
% contiguous subranges of a vector,
% called \textit{partial vectors} (\texttt{pvector}s), 
% which can be passed as \texttt{vector}-compatible
% operands to a function instead of an entire \texttt{vector}.
% In this way, it is possible to pass several (disjoint or even
% overlapping) \texttt{pvector} objects as seemingly different \texttt{vector}
% arguments that however are just windows onto
% a common \texttt{vector} container variable. 




% CK new 190127 text from LL:

\subsection{Partial Vectors}
%{{{
\label{sec:Partial Vectors}

% first paragraph replaced by old pvector description below:
%To cope with functions that operate on only a regular part of a normal VectorPU vector
%efficiently, VectorPU provides a specialized kind vector called partial vector,
%that can represent a partial range of a normal vector.

% New 180916 CK:
Using iterators, it is possible in VectorPU to define references to
contiguous subranges of a vector,
called \textit{partial vectors} (\texttt{pvector}s), 
which can be passed as \texttt{vector}-compatible
operands to a function instead of an entire \texttt{vector}.
In this way, it is possible to pass several (disjoint or even
overlapping) \texttt{pvector} objects as seemingly different \texttt{vector}
arguments that however are just windows onto
a common \texttt{vector} container variable. 
In contrast to \texttt{vector}s without \texttt{pvector}s, where coherence is 
managed automatically by VectorPU, the coherence 
management in the presence of \texttt{pvector}s 
is exposed to the programmer.

A partial vector can be initialized by two iterators to a normal VectorPU \texttt{vector} (we call it \emph{mother vector}).
% representing the convex closure of the index ranges of
% all contained partial vectors. 
No new memory is allocated for this partial vector, 
only the two iterators are stored, 
and the coherence state for its range in the mother vector.
% for the range of data this partial vector contains.
When a partial vector is declared, it automatically
inherits the coherence state information
from its mother vector. 
% already said above:
% Afterwards, VectorPU algorithms can operate on the initialized partial vector
% in the same way as for other VectorPU vectors.

\begin{figure}
\noindent 
\begin{minipage}{\linewidth}
\begin{footnotesize}% changed from small to have same font size as in other code snippets
\begin{verbatim}
struct my_set {
   template <class T>
   __host__ __device__
      void operator() (T &x) { x+=101; } 
};
vectorpu::vector<int> vx(10);  // the mother vector
vectorpu::pvector<int> vy(x, vx.begin(), vx.begin()+2);
vectorpu::for_each<int>( GWI(vy), GWEI(vy), my_set() );
vectorpu::for_each<int>( GWI(vy), GWEI(vy), my_set() );
SR( vy );  // explicit coherence management
vectorpu::for_each<int>( RI(vx), REI(vx), [](auto x) {cout<<x<<" ";} );  
\end{verbatim}
\end{footnotesize}
\end{minipage}
\caption{\label{fig:pvector}Example of using a partial vector (\texttt{pvector}) and the \texttt{SR} macro for explicit coherence management. (Note: \texttt{pvector} is a short form and actually called \texttt{parco\_vector} in the VectorPU API.)}
\end{figure}

% One important point to make this mechanism work is for the case 
The aliasing introduced by \texttt{pvectors} can lead to coherence problems. One such scenario could be that the
programmer intends to operate on the previous \texttt{vector} again after
some part of it was updated via a \texttt{pvector}, 
hence the whole \texttt{vector} would be in an inconsistent state.
In such cases, VectorPU expects the programmer to use a macro \texttt{S$X$} (\texttt{S}ynchronize for access mode $X$,
such as \verb.SR. for synchronized read)
just before the programmer operates on the whole vector again.
It may be inefficient for a \texttt{pvector} to perform the \texttt{S$X$} synchronization
automatically, because multiple operations can be performed on the same \texttt{pvector} 
before accessing the whole \texttt{vector} again, 
and because the \texttt{pvector}
has no knowledge about when the operations on itself will be finished, hence
keeping them coherent each time is not necessary and thus a waste of performance.

% Listing~\ref{lst:parco_vector} 
Figure~\ref{fig:pvector}
shows an example of using a \texttt{pvector} and the 
\texttt{S$X$} macro.
It initializes a mother vector \texttt{vx}
and a \texttt{pvector} \texttt{vy} on it.
The following two lines change part of \texttt{vy}'s value multiple times. % by operating on the \texttt{pvector}
%using the functor in Listing~\ref{lst:skeleton_programming}.
The \texttt{SR} macro explicitly restores coherence for
\texttt{vy} before the following read access, 
resulting in a write-back of \texttt{vy} elements in GPU device memory
to their locations in \texttt{vx},
% only once for the multiple changes partially, 
thus also \texttt{vx} as a whole becomes coherent again
and line 6 is safe to operate on the whole \texttt{vx}.

% CK: I do not understand this, dropped for now.
% For applications in which the workload changes dynamically from one iteration to another,
% e.g.\ a parallel reduction on GPU where the workload 
% shrinks by half at each iteration,
% it is possible to initialize \texttt{pvector}s dynamically.
% Hence, when switching to a CPU implementation 
% as the workload is small enough,
% the partial data get coherence-managed automatically
% by VectorPU.
%}}}



\begin{figure}[tb]
\begin{minipage}{.48\textwidth}
\begin{small}
\begin{verbatim}
void coherent_on_cpu_r(){
  	 if( !cpu_coherent_unit ){
       download();
       cpu_coherent_unit=true;
  	 }
}
void coherent_on_cpu_w(){
  	 cpu_coherent_unit=true;
  	 gpu_coherent_unit=false;
}
void coherent_on_cpu_rw(){
  	 if( !cpu_coherent_unit ){
       download();
       cpu_coherent_unit=true;
  	 }
  	 gpu_coherent_unit=false;
}
\end{verbatim}
\end{small}
\end{minipage}\qquad~
\begin{minipage}{.48\textwidth}
\begin{small}
\begin{verbatim}
void coherent_on_gpu_r(){
  	 if( !gpu_coherent_unit ){
       upload();
       gpu_coherent_unit=true;
	 }
}
void coherent_on_gpu_w(){
  	 gpu_coherent_unit=true;
  	 cpu_coherent_unit=false;
}
void coherent_on_gpu_rw(){
  	 if( !gpu_coherent_unit ){
       upload();
       gpu_coherent_unit=true;
  	 }
  	 cpu_coherent_unit=false;
}
\end{verbatim}
\end{small}
\end{minipage}
\caption{\label{fig:vectorpucoherence}Coherence control code,
    here for simple vectors,
    in \texttt{vectorpu.h}. Functions \texttt{download} and
    \texttt{upload} are implemented using CUDA
    \texttt{thrust::copy}. Validity of copies on CPU and GPU
     is indicated
     by the flags \texttt{cpu\_coherent\_unit} and
     \texttt{gpu\_coherent\_unit} respectively; both are
     initialized to \texttt{true}
     in code allocating new vectors (not shown).}
     % 
     % for the records:
     %
     % template <class T, class Index_Type=std::size_t>
     % struct min_vector : public std::vector<T>, public thrust::device_vector<T> {
     % explicit min_vector(Index_Type _array_size):
     %  	 std::vector<T>::vector(_array_size),
     %  	 thrust::device_vector<T>::device_vector(_array_size),
	 % cpu_coherent_unit(true),
	 % gpu_coherent_unit(true),
     %  	 array_size(_array_size),
     % start_pos(0) {}
\end{figure}




%\vspace{1.4mm}
%\noindent
%\textit{Implementation notes }
\subsection{Implementation Notes}

\paragraph{Coherence protocol}
%
In the source code of VectorPU, the code
relevant for our work is the part of \verb+vectorpu.h+%
\footnote{The VectorPU source code can be found at
\texttt{https://github.com/lilu09/vectorpu}.} 
that handles coherence. Its implementation for the various
variants of \texttt{vector} 
(see the code excerpt in Fig.~\ref{fig:vectorpucoherence} for
simple \texttt{vector}s) follows a
simple valid-invalid protocol.
%(single-reader single-writer sharing).

% CK new 190127 - text snippet from LL:
\paragraph{Expansion of macro annotations to device-specific pointers}

For function parameters, the macro annotations expand into appropriate C code to fit their function call context.
For illustration, we show the simplified code after a function parameter's expansion ($\longrightarrow$) for four typical annotations,
where \texttt{vx} refers to a VectorPU \texttt{vector} instance:

\begin{small}
\begin{itemize}
 \item \texttt{R(vx)} $\longrightarrow$ 
 \begin{minipage}[t]{0.8\textwidth}
  \texttt{set\_coherence\_state();}\\
  \texttt{return this->std::vector<T>::data();}\\
  \texttt{//casted as const in return value}
 \end{minipage}
 \item \texttt{W(vx)} $\longrightarrow$ 
 \begin{minipage}[t]{0.8\textwidth} \texttt{set\_coherence\_state();}\\
 \texttt{return this->std::vector<T>::data();}
 \end{minipage}
 \item \texttt{GR(vx)} $\longrightarrow$ \begin{minipage}[t]{0.8\textwidth}
 \texttt{set\_coherence\_state();}\\
 \texttt{return thrust::raw\_pointer\_cast(}\\
 \hspace*{1em}\texttt{\& (* thrust::device\_vector<T>::begin() ) );}\\
 \texttt{//casted as const in return value}
 \end{minipage}
 \item \texttt{GW(vx)} $\longrightarrow$ 
\begin{minipage}[t]{0.8\textwidth} \texttt{set\_coherence\_state();\\ return thrust::raw\_pointer\_cast(}\\
\hspace*{1em}\texttt{\& (* thrust::device\_vector<T>::begin() ) );}
\end{minipage}
\end{itemize}
\end{small}

Hence, each annotated parameter is expanded to some code snippet\footnote{One can think of those code snippets as
anonymous functions or lambda functions. 
In the real scenarios these code snippets are encapsulated within a function,
and each macro as shown above expands to a call to its function.}.
In all scenarios the expanded macros first update the coherence state according to the annotation's semantics.
Then, for the CPU cases, a pointer to a \texttt{std::vector} is returned,
and for the GPU cases, a pointer to a Thrust pointer (which is a pointer to GPU memory space) is returned.
For read-only cases, the return value is casted to \texttt{const} to ensure type safety in its function invocation.
For write-only cases, no such \texttt{const} cast happens.

\paragraph{Partial vector implementation and memory management}
% new 180916 CK:
For implementing the \texttt{pvectors} atop \texttt{vector}s, 
VectorPU uses the simplistic approach of allocating memory for
the \emph{entire} \texttt{vector} on the device 
even if the \texttt{pvector}(s) might only
access a minor part of it. This may waste device memory space 
but makes local address calculations easy, and anyway only
the accessed elements (the \texttt{pvector} range) will be transferred.
As we will see later, it also simplifies coherence management
for \emph{overlapping} \texttt{pvector} accesses, which was not 
really foreseen 	in the original VectorPU design.

% CK new 190127 input from LL:
In order to better utilize the device memory,
a future extension of VectorPU could implement 
\emph{lazy allocation}.
In lazy allocation mode, VectorPU maintains for each 
\texttt{vector} another state
flag called \texttt{is\_allocated}
to track whether it is allocated for host/device memory or not.
When a VectorPU \texttt{vector} is declared,
by default, no device memory is really allocated and the \texttt{is\_allocated} flag is set to false.
Only upon its first use in a function call as an
annotated \texttt{vector} argument,
the flag \texttt{is\_allocated} is checked.
If it is false (not allocated), device memory is allocated,
otherwise, the memory is already allocated and no new memory will be allocated.
Afterwards, the function call continues.

%\vspace{1.4mm}
%\noindent
%\textit{Efficiency }
\subsection{Efficiency}

Using only available C++(11) language features, 
VectorPU provides a flexible unified memory
view where all data transfer and device memory management
is abstracted away from the programmer. Nevertheless,
its efficiency is on par with that of handwritten CUDA code
containing explicit data movement and memory management code
\cite{VectorPU-2017}.
In particular, the VectorPU prototype was shown to
achieve 1.4x to 13.29x speedup over good quality
code using Nvidia's \textit{Unified Memory} API
on several machines ranging from laptops to supercomputer nodes,
with Kepler and Maxwell GPUs. For a further discussion of
VectorPU features, such as
specialized versions of \verb-vector-, for descriptions
of how to use VectorPU together with lambda expressions 
e.g.\ to express skeleton computations, and for further
experimental results 
we refer to \cite{VectorPU-2017}.


\section{A Formalization for Reasoning on  Consistency in VectorPU}\label{sec:Formal}

In this section we provide a minimal calculus to reason on the memory operations that can 
exist in a framework that deals with memory consistency like VectorPU. We first define a 
set of effects that operations can have on the consistency of the memory. Then we define a 
small calculus expressing different memory accesses and their composition into complex 
procedures. Finally, we express VectorPU operations as higher-level statements that can 
be translated  into the core calculus, and show that if all memory 
accesses are annotated correctly through VectorPU annotations the program cannot try to 
access an invalid data and the memory spaces are put in coherence when needed. We also 
show that VectorPU tracks the validity status of the memory adequately. In this 
section we abstract away the values stored in memory and we 
do not deal with any form of aliasing. A more precise analysis of effects and aliasing is 
out of the scope of this paper, it could be for example inspired 
from~\cite{Nielson1999}.
We place ourself in a simplified setting where each variable is hosted in exactly two 
memory locations, e.g.\ a CPU (main) memory and a GPU memory location, but the work could be extended to multiple memory 
locations without any major difficulty.

\subsection{An effect system for consistency between memory locations}
We start from a simple effect system, it expresses the effect of writing or reading a 
memory 
location on the consistency status of the memory. Each location is either in  
\textit{valid} state when it holds a usable data 
or \textit{invalid} state when the value at the location is not valid anymore.

We express five  operations: reading, writing, 
\Push\ for uploading the local memory location into the other one, and \Pull\ for the 
contrary. \noop\ is an operation that does nothing.
 \[E::= \Push ~|~ \Pull ~|~ r~|~ w ~|~ %rw ~|~ 
\noop\]

The effect of these operations express their requirements and effects on a single memory 
location. We 
express 
below the semantics of each of the operations on the consistency status of the concerned 
memory location. 
%The status memory location 
The \textit{memory status of a variable}
is a pair of the status of its locations, 
where each status is 
either $V$ for valid or $I$ for invalid. The first element is the status of the local 
memory, and the second one is the status of the remote memory. For example, for a 
program running on a CPU while the remote memory is a GPU, a status $(V,I)$ means that 
the memory is valid and can be read on the CPU, but is invalid on the GPU and should be 
transferred before being usable there.

Each operation has a signature in the sense that it may require a certain memory status 
and 
will produce another memory status. The signature of each operation is expressed below.
We use
variables $X$, $Y$, $Z$, $T$ that are considered as universally quantified in each rule. 
They can 
be instantiated with either $V$ or $I$.
\begin{mathpar}
\Push: (V,X)\mapsto (V,V)

\Pull: (X,V)\mapsto (V,V)

r: (V,X)\mapsto (V,X)

%rw: (V,X)\mapsto (V,I)

w: (X,Y)\mapsto (V,I)

\noop: (X,Y)\mapsto (X,Y)
    \end{mathpar}

These signatures are effects  expressing that
$r$ is a reading operation requiring validity of data and ensuring not to modify it, 
the distant status is unchanged; $w$ 
is a  writing operation that modifies data locally but do not require validity, it 
invalidates the remote memory.  \Push\ uploads the local memory and thus makes valid the 
distant memory; 
it requires that the data is locally valid, and \Pull\ is the symmetrical operation.

An additional operation could be defined: 
 an 
$rw$ operation would represent a read and/or write access, it would both require data 
validity and invalidate 
remote status: $(V,X)\mapsto (V,I)$. This operation is however not needed here but we will have a similar one at the annotation level, see below.

\subsection{A language for modelling consistency and effects}\label{sec-core}
We now create a core calculus to be able to reason on programs involving sequences of 
effects on different memory locations. $x,y$ range over variables and we introduce  statements manipulating 
variables. We use sequence and simple loops and conditionals. 
Operations with effects 
now apply to a variable;  the $\rem{E~x}$  is a remote operation 
 on the remote memory. For example, a GPU procedure 
writing $x$ and reading $y$ would correspond to the pseudo-code: $\rem{w~x};\rem{r~y}$. 
Statements $S$ are defined as:

{\small \[S::=E~x~|~\rem{E~x}~|~S;S'~|~\while(\cond) S ~|~ \IF{\cond}{S}{S'}\]}
\noindent where $E$ $x$ denotes some effect $E$ on variable $x$, with $E\in \{r,$ $w,$  
$\Push,$ $\Pull,$ $\noop\}$.

We are  interested  in conditionals dealing 
with the validity status of the variables. Other conditionals  are expressed as a generic 
binary operator $\oplus$ but  operators with different arities could be added as 
well:
\[\cond::=\isvalid~x~|~\isremvalid~x~|~x\oplus y\]
%$\oplus$ replaces operations on values; we are not interested in evaluating them ; only 
%to show that while can use values (perhaps to be improved)
\noindent where \textit{isValid} $x$ and \textit{remIsValid} $x$ denote checks of the validity status flag of the local and remote location of $x$, respectively.

We now define a small step operational semantics for our core calculus.
It relies on the validity status of variables, recorded in a store $\sigma$ mapping 
variable names to validity pairs. Semantics is written as a transition relation between 
pairs consisting of a statement and a store: $(S,\sigma)$. The sequencing operator $;$ is associative with \noop\ as a neutral element. 
Consequently each non-empty sequence of instruction can be rewritten as $S;S'$ where $S$ 
is neither a 
sequence nor \noop. $\sigma[x\mapsto (X,Y)]$ is the update operation on maps. 
%The swap 
%operation on 
%pairs is extended to store (it swaps the value associated to each variable)


\begin{figure*}[tp]
\begin{mathpar}
\inferrule[valid]
{\sigma(x)=(V,X)}
{\feval{\isvalid~x}{\sigma}=\True}

\inferrule[invalid]
{\sigma(x)=(I,X)}
{\feval{\isvalid~x}{\sigma}=\False}

\inferrule[rem-valid]
{\sigma(x)=(X,V)}
{\feval{\isremvalid~x}{\sigma}=\True}

\inferrule[rem-invalid]
{\sigma(x)=(X,I)}
{\feval{\isremvalid~x}{\sigma}=\False}

\inferrule[Effect]
{\sigma(x)=(X,Y) \\ E:  (X,Y)\mapsto (Z,T)  }
{(E~x;S,\sigma)\to (S,\sigma[x\mapsto(Z,T)])}


\inferrule[Remote Effect]
{\sigma(x)=(X,Y) \\ E:  (Y,X)\mapsto (Z,T) }
{(\rem{E~x};S,\sigma)\to (S,\sigma[x\mapsto(T,Z)])}

\inferrule[While-True]
{\feval{\cond}{\sigma} }
{(\while(\cond) S;S',\sigma)\to (S;\while(\cond) S;S',\sigma)}

\inferrule[While-False]
{\neg\feval{\cond}{\sigma} }
{(\while(\cond) S;S',\sigma)\to (S',\sigma)}

\\

\inferrule[IF-True]
{\feval{\cond}{\sigma} }
{((\IF\cond S S');S'',\sigma)\to (S;S'',\sigma)}

\inferrule[IF-False]
{\neg\feval{\cond}{\sigma} }
{((\IF\cond S S');S'',\sigma)\to (S';S'',\sigma)}

    \end{mathpar}
\caption{Operational semantics of validity status.}\label{fig:Opsem}
\end{figure*}

The semantics is presented in Figure~\ref{fig:Opsem}. Like in the previous section, we 
use validity variables $X$, $Y$, $Z$, $T$ 
that are universally quantified in each rule.
 The first %three
 four rules present the 
evaluation of conditional statements, we suppose additional rules exist for evaluating 
$\oplus$\footnote{We are only interested in cache consistency properties, we thus suppose  that evaluation of $\oplus$ always succeed, and in particular  variables accessed by the operation are specified as a $r$ operation preceding the condition.}. The next rule applies an effect on a variable $x$ updating the validity store, 
and the \textsc{Remote Effect} rule applies an effect occurring on the distant memory, it 
applies the symmetric of the effect to the variable. Note that \Push\ is the symmetric of 
\Pull\ and we could have removed one of those two operations without loss of generality. 
The last rules are standard ones for \symb{if} and \symb{while} statements.

\noindent\emph{Initial state:} To evaluate a sequence of statements $S$ using the 
variables 
$\vars(S)$, we put it in a configuration with 
an initial 
store where data is hosted on the CPU and all variables are initially mapped to $(V,I)$: 
$\sigma_0=(x\mapsto 
(V,I))^{x\in \vars(S)}$.


%for the moment 2 address spaces, see if we need many of them

A configuration is \emph{reachable} if it is possible to obtain this configuration 
starting from the initial configuration and applying any number of reductions: 
$(S,\sigma)$ is reachable if $(S,\sigma_0)\to^*(S',\sigma)$ where $\to^*$ is the 
reflexive 
transitive closure of $\to$. We write $(S,\sigma)\not\to $ and say that the configuration 
is \emph{stuck} if no reduction rule can be 
applied on $(S,\sigma)$.
\begin{Property}[Progress]\label{prop:stuck}
 A configuration is stuck if the validity status of the accessed variable is 
incompatible with the effect to be applied\footnote{We say that there is no unification  between $X$ and $Y$ if one of the two variables must have the value $V$, and the other one the value $I$. This relation is extended to pairs of variables.}:
\[\begin{array}{@{}l@{}}
(S,\sigma)\not\to\ \iff~
 \begin{array}[t]{@{}l@{}}
S=E~x;S' \land \sigma(x)=(X,Y) 
							\land E:  (X',Y')\mapsto (Z,T)\ 
						\\\qquad 	\land~\text{there is no unification  between } 
							(X,Y)   
							\text{ and } (X',Y')\\
\lor S\!=\!\rem{E~x};S' \land \sigma(x)\!=\!(X,Y) 
							\land E:  (X',Y')\!\mapsto\! (Z,T)\ 
							\\\qquad \land~\text{there is no unification  between } 
							(X,Y)   
							\text{ and } (Y',X')		
\end{array}
\end{array}\]
Note that this supposes that $\oplus$ always succeeds.
\end{Property}
\begin{proof}[Proof sketch]
By case analysis on the first statement of $S$, there is always 
one rule applicable provided the premises of the rule can be evaluated.
In the case of the last four rules this requires the evaluation of \cond. If $\oplus$ 
always succeeds then \cond\ can always be 
evaluated. The only case remaining is if there is no unification possible between the 
effect of an operation and the current validity status of the affected variable, this 
concerns the rule \textsc{Effect} and \textsc{Remote Effect} and corresponds to the two 
cases expressed in the theorem.
\end{proof}


\begin{Property}[Safety]\label{prop:safe}
A state is said to be \emph{unsafe} if at least one variable is mapped to 
$(I,I)$.
It is impossible to reach an unsafe state from the initial state.
\end{Property}
\begin{proof}[Proof sketch] 
Unsafe states are avoided  because of the effects of operations: 
only effect rules modify the store and no effect can reach $(I,I)$, except 
\noop\ starting from $(I,I)$, whic is sufficient to conclude as the inital state is not $(I,I)$.
\end{proof}

\noindent
   \emph{Example:} 
    $w x;\rem{r~x}$ cannot be fully evaluated. Indeed, $(w x;\rem{r~x},(x\mapsto 
    (V,I)))\to (\rem{r~x},(x\mapsto (V,I)))$, but  $\rem{r~x}$ requires that $x$ is 
    mapped to $(X,V)$ for some $X$ which is not the case.
However if we add a \Push\ operation to ensure the validity of the accessed memory the 
program $w~x;Push;\rem{r~x}$ can be reduced as follows:\\
$(w x;\rem{r~x},(x\mapsto (V,I)))\\~\qquad\to (\Push;\rem{r~x},(x\mapsto 
(V,I)))\\~\qquad\to 
(\rem{r~x},(x\mapsto (V,V)))\to (\noop,(x\mapsto (V,V)))$

%\subsection{Protecting Memory Accesses}
%We now define protected operations as macros that can be compiled into the core 
%calculus. 
%The objective of this section is to show that if each memory access is protected, i.e.\ 
%if 
%a program only contains protected effects, and no direct memory effect, then the program 
%executes safely without getting stuck in a configuration. We propose an extended 
%language 
%with the following syntax:
%\[S'=S~|~R~x~|~W~x~|~RW~x|~\rem {R~x}~|~\rem {W~x}~|~\rem {RW~x}\]
%For each memory effect, we now have a protected access; for example, $R~x$ denotes a 
%protected reading operation that ensures that the validity status is correct before 
%performing the read. We have protected local accesses and protected accesses on the 
%distant memory. Similarly to the VectorPU library, the protected accesses can be 
%considered as macros and the programs of the extended syntax can be translated into the 
%core syntax as follows:
%\begin{mathpar}
%\transl{R~x}=(\IF{\isvalid~x}{\noop}{\Pull~x});r~x
%
%\transl{\rem {R~x}}=(\IF{\isremvalid~x}{\noop}{\Push~x});\rem{r~x}
%
%\transl{RW~x}=(\IF{\isvalid~x}{\noop}{\Pull~x});rw~x
%
%\transl{\rem{RW~x}}=(\IF{\isremvalid~x}{\noop}{\Push~x});\rem{rw~x}
%
%\transl{W~x}= w~x
%
%\transl{\rem{W~x}}= \rem{w~x}
%\end{mathpar}
%This encoding corresponds  to the macros instantiated in VectorPU except that VectorPU 
%additionally tracks the effects, and this is unnecessary here because our semantics is 
%tracking the effect. Note that it is easy to check that VectorPU tracks the effects in 
%the same way as our semantics does. These translation rules  perform \Push\ or 
%\Pull\ operations in order to ensure that the memory is in a correct validity status for 
%the read or write operation to be performed.
%
%\begin{Theorem}[Protecting operations ensures progress]
%If a program $S'$ is written with only  protected operations, i.e. no $r$, $w$, $rw$, 
%$\Push$ or $\Pull$, then its execution cannot reach a stuck configuration.
%\end{Theorem}
%
%\begin{proof}[Proof sketch] By Property~\ref{prop:stuck}, it is sufficient to prove that 
%unification on the validity status is always possible, in other words, if a program has 
%only protected operations then this unification is always possible. 
%
%This is done by case analysis of the effect applied in the translated program, and 
%showing that in each case the effect rule can be applied.
%Take the example of \Pull. \Pull\ only appears upon $R~x$ or $RW~x$ and when  
%$\isvalid~x$ is 
%false. In this case $\sigma(x)=(I,V)$ by property~\ref{prop:safe}. Consequently \Pull\ 
%is 
%not stuck.
%Now considering the effect $r$, $r~x$ can only be obtained by translation of $R~x$. 
%Consequently, $r~x$ appears necessarily after a conditional \Pull, thus when we reach 
%$r~x$ we necessarily have $\sigma(x)=(V,I)$ or 
%$(V,V)$ (by a trivial case analysis of the possible validity status of $x$ before the 
%\Pull). 
%Other cases are similar.
%\end{proof}


\subsection{Declaring access modes and adding an abstraction layer}
The calculus defined above only considers simple memory locations and directly manipulates 
them.
But VectorPU and  similar libraries manipulate structures 
representing the memory. For example, VectorPU vectors act as an
 abstract representation of a set of memory 
locations. In this section, we add a declaration and abstraction layer to the calculus to 
represent the access mode declarations that will trigger data transfers according to the 
consistency mechanism. 
This abstraction layer is also a necessary first step to the modelling of array 
structures that we will present in Section~\ref{sec-arrays}. Indeed, in array structures, 
the 
validity status of the array is abstracted away by a single validity status pair. Then 
%approximations inspired from classical abstract interpretation techniques~\cite{Cous77} 
a dynamic abstraction of the consistency status of the memory can be used.
%can be used except that the abstract view is computed dynamically but must stay 
%consistent with the validity status of the real memory.
More technically, the abstraction and declaration layer relies on two principles:
\begin{itemize}
\item Each variable $x$ has an abstract variable $\abs x$ that represents it. In this 
section there is 
a single variable for each representative, but when we deal with arrays we will 
have a single representative for the whole array.
\item It is safe to ``forget'' that one memory space holds a valid copy of the data if 
the other memory space has a valid one. In other words, $(V,I)$ (resp. $(I,V)$) is a safe 
abstraction of $(V,V)$ and we denote $(V,I)\leq (V,V)$ (resp. $(I,V)\leq (V,V)$). Of 
course, we have $(X,Y)\leq (X,Y)$ for any $X$ and $Y$.
\end{itemize}

\paragraph{Syntax}
We now define access mode declarations:\\[-3.3ex]
\begin{align*}
\AM&::=R\ \abs x \,|\, W\ \abs x \,|\, RW\ \abs x \,|\, 
\rem{R\ \abs x} \,|\,\rem{W\ \abs x} \,|\,\rem{RW\ \abs x} \,|\, \\
&~ \AM \land \AM' \quad \text{(where variables in $\AM$ and $\AM'$ are disjoint)}
\end{align*}

These access modes declare the kind of access (read $R$, write $W$, or read and/or write 
$RW$) that 
can be performed on the variable $x$ represented by $\abs x$. In a set 
of access mode declarations the same variable cannot appear twice. There exist declared 
access modes for  local accesses and for  the 
remote memory space.

% added "calls to" - one could even use the term "components" from Sect. 2
A program is a sequence of calls to functions or components (i.e., statements accessing 
only real variables) 
each protected by an access 
mode declaration (on abstract variables representing the real variables):
\[\Prog::=\AM_1\{S_1\};\AM_2\{S_2\};\ldots\]
We write  $S\in S'$ if $S$ is one statement inside $S'$ (i.e. $S$ is a sub-term of 
$S'$).

We  define below the semantics of these programs  and specify well-declared program by 
comparing the statements they contain with 
the declared access modes. The semantics relies on the translation of 
the access mode declarations into consistency mechanisms 
with checks and data transfers 
triggered 
before each function 
execution.


\paragraph{Extension of statements to abstract variables}
When evaluating a program, the store contains both real and abstract variables, and the 
existing 
statements have the same effect on the abstract variables as on the real ones. However 
one should notice that 
even if the effect is the same, the meaning of a statement acting on a real variable 
or on its representative is different: in our calculus, the effect on a variable is an 
abstraction of the real effect that involves side effects and data transfers. On the 
contrary, only the validity status of abstract variables is stored by the library: the 
effect triggered by an operation on an abstract variable is exactly what happens when 
VectorPU updates the validity status of its internal structures.

For example, a \Pull\ operation on a real variable consists in transferring data from a 
remote memory space 
to the local one. We abstracted it  by changing the local validity status. A \Pull\  
operation on an abstract variable only changes the validity status, no data transfer has 
to be done because abstract variables only need to be stored in one memory space. 
The validity status is stored in the CPU address space in VectorPU. Comparing the validity 
status of real memory and their representative  allow us to reason 
formally on the correctness of the validity tracking performed by VectorPU.

\begin{figure*}[tb]
\begin{mathpar}
\transl{R~\abs x}=(\IF{\isvalid~\abs x}{\noop}{(\Pull~x;\Pull~\abs x)})

\transl{\rem {R~\abs x}}=(\IF{\isremvalid~x}{\noop}{(\Push~x;\Push~\abs x)})

\transl{RW~\abs x}=(\IF{\isvalid~x}{\noop}{(\Pull~x;\Pull~\abs x)});w~\abs x

\transl{\rem{RW~\abs x}}=(\IF{\isremvalid~x}{\noop}{(\Push~x;\Push~\abs x)});\rem{w~\abs 
x}

\transl{W~\abs x}= w~\abs x

\transl{\rem{W~\abs x}}= \rem{w~\abs x}

\transl{\AM_1\{S_1\};\AM_2\{S_2\};\ldots} = \transl{\AM_1};S_1;\transl{\AM_2};S_2;\ldots
\end{mathpar}
\caption{Semantics of access modes and programs}\label{sem-AM}
\end{figure*}
As no data is accessed by the effects on abstract variables, they cannot create stuck configuration. Consequently, $r~\abs x$ has no effect as it does not 
change the validity 
% Comment: This is only because VectorPU uses the most primitive
% cache coherence protocol, the MI (VI) protocol.
% More elaborated coherence protocols like MSI or MESI introduce additional
% states where also reads trigger state transitions to follow up the
% number of readers (one or larger than one).
status of variables. The statement that should get stuck in case of a read access is the 
read of \emph{the real variable that cannot access a valid data}. 
%Similarly, the interesting 
%effect of read/write ($rw$) operations is entirely represented by the write ($w$) access 
%and $rw$ is not needed over abstract variables.


\paragraph{Semantics}

Figure~\ref{sem-AM} defines the semantics of programs with access modes as a translation 
into the core calculus of Section~\ref{sec-core}. 
This translation
ensures that the validity status is correct and records the effect of the function on the 
abstract variable before running the function call that may 
read and write data (on the real variables).  Similarly to the VectorPU library, the 
protected accesses can be 
considered as macros and the programs can be translated into the 
core syntax.

This encoding corresponds  to the macros as they are implemented in VectorPU.  
It is indeed easy 
to check that VectorPU tracks the effects in 
the same way as our effect system  does in the translation rules. These translation 
rules  perform \Push\ or 
\Pull\ operations in order to ensure that the memory is in a correct validity status for 
the read or write operation to be performed.
When evaluating a program we create a store where the validity status of real and 
abstract variables are $(V,I)$, corresponding to the fact that data is 
initially placed in one memory location; typically, in VectorPU, in the CPU memory space.
%
%
%From this semantics, we state two crucial correctness properties for the library and 
%show 
%their correctness in a semi-formal way.


\subsection{Well-declared Programs and their Properties}
We now define formally what it means for an access mode declaration to be correct, i.e. 
to adequately specify the effect of a function. The principle is that each operation on a 
memory location must be declared on its representative. It is however possible to declare 
more read or $RW$ accesses that what is done in practice, and one can declare a read 
and/or write 
access if only read or write is performed. Additionally, the annotation $W$ denotes an 
\emph{obligation} to write which allows the consistency mechanism to avoid any validity 
check and any transfer before running the function that will overwrite the data. 
To represent this concept, we need a first definition that states that an operation will be performed in all execution paths of a (bigger) statement. This 
definition 
formalises a classical static analysis concept that states that all branches of 
conditionals necessarily execute a given statement. It considers executions that run to 
completion and states that a given statement is necessarily evaluated in this execution.

\begin{definition}[Occur in all execution paths]
We state that a statement\emph{ $S$ occur in all execution paths of $S_0$} if, for any 
correct 
initial store $\sigma_0$, for all full reductions 
$(S_0,\sigma_0)\to(S_1,\sigma_1)\to\ldots\to(\noop,\sigma_n)$, there is an intermediate 
state $(S_i,\sigma_i)$ such that $S_i=S;S''$ for some $S''$.
\end{definition}
Notice that an operation $S$ can only appear in some of the execution paths of $S'$ if  $S\in 
S'$: if $(S_0,\sigma_0)\to^* (S;S',\sigma)$ then $S\in S_0$.


\begin{definition}[Well-declared program]\label{def-WD}
A program $\Prog$ is \emph{well-declared} if for all $\AM\{S\}$ in $\Prog$ we have:
\begin{itemize}
\item $\Push\ x\not\in S$ and $\Pull\ x\not \in S$ (for any $x$),
\item $w\ x\in S \implies (W\ \abs x \in \AM \lor RW\ \abs x \in \AM)$,
\item $r\ x\in S \implies (R\ \abs x \in \AM \lor RW\ \abs x \in \AM)$,
\item $W\ \abs x\!\in\! \AM \!\implies\! w\ x$ occurs in all execution paths of $S$,
\item Plus the same rules for remote operations.
\end{itemize}
\end{definition}
Note that a well-declared program does not perform synchronisation 
operations (\Push\ or \Pull) manually, these operations are only  performed when 
evaluating the 
access mode declarations. Also each variable accessed by a well-declared function has an 
abstract representative in the corresponding declaration block.


A direct consequence of the definition above is that a well-declared program cannot 
access, in the same function, the same
variable in both address spaces. This is in accordance with VectorPU where each function 
is entirely executed either on a CPU or on a GPU, the formalisation is a bit more 
 generic on this aspect. This is expressed by the following property.
\begin{Property}[Localised access]\label{prop-localised}
For a well-declared program containing $\AM\{S\}$, for any $x$, we cannot have $\rem {E\ 
x} \in S$ and $E'\ x \in S$.
\end{Property}

\smallskip

We now state and  prove the two major properties ensured by our formalisation.
The first property ensures that the abstraction is correct relatively to the 
execution. This corresponds to the fact that VectorPU tracks adequately the validity 
status of the 
memory. This is expressed as a theorem that is similar to subject-reduction in type 
systems, it states that if the status of the abstract variables represent correctly the 
validity status of the real variables, then the 
abstraction is also correct after the execution of a  well-declared function.  Let us say 
that  we have a \emph{correct abstraction of the memory state} if for each real memory 
location, the abstract representative of this location has a validity status that is an 
approximation, in the sense of $\leq$, of the validity status of the real memory. The 
theorem below states that the execution of a well-declared function maintains the 
correctness of the memory state 
abstraction. 

\begin{Theorem}[Subject reduction]\label{thm-SR}
Suppose $\AM\{S\}$ is well-declared, we have:
\[\begin{array}{@{}l@{}}
\forall x\in \dom(\sigma).\, \sigma(\abs x)\leq\sigma(x) 
\land
 (\transl{\AM\{S\}},\sigma) \to^* (\noop,\sigma')
\\~~~~~
\implies \forall x\in \dom(\sigma').\, \sigma'(\abs x)\leq\sigma'(x) 
\end{array}\]

This property is extended by a trivial induction to the execution of a well-protected 
program $\Prog$ in an initial store $\sigma_0=(x\mapsto 
(V,I))^{x\in \vars(\Prog)}$.
\end{Theorem}
\begin{proof}
Notice that $\dom(\sigma')=\dom(\sigma)$, and  if $\sigma(x)=(V,I)$ or $\sigma(x)=(I,V)$ 
then $\sigma(x)=\sigma(\abs x)$, else $\sigma(x)=(V,V)$.
We reason on the read and write access that occur in the considered reduction. Each 
variable $x$ is either read or written or not accessed (or read and written). For each 
case we compare the 
status of abstract and local variable, and in particular we consider the status of the 
reduction after executing the synchronisation code $\transl{\AM\{S\}}$ and call 
$\sigma_s$ the corresponding store (note that $\sigma_s(\abs x)=\sigma'(\abs x)$). We 
detail operations on the local 
address space, cases for remote operations are similar:

\noindent$\bullet$ If  $x$ is written, we have:
$(\transl{\AM\{S\}},\sigma)\to^* (w~x;S',\sigma'')  \to^* (\noop,\sigma')$. Whatever the 
initial value of $\sigma(x)$, we have $\sigma'(x)=(V,I)$. Two cases are possible:

\noindent$(1)$ $W~\abs x \in \AM$ then the value cannot be read and we have 
$\sigma_s(\abs x)=(V,I)$.  $ \sigma'(\abs x)=\sigma'(x)$.

\noindent$(2)$ $RW~\abs x \in \AM$ then a data-transfer (\Pull) may occur. Knowing that 
$\sigma(\abs x)\leq\sigma(x)$, by a  
case 
analysis on $\sigma(x)$ and $\sigma(\abs x)$ we have: $\sigma_s(\abs x)=(V,I)$ and 
$\sigma_s(x)=(V,I)$ or $(V,V)$. Whether $x$ is  read or not we have $ \sigma'(\abs 
x)=\sigma'(x)$.

\noindent$\bullet$ If  $x$ is read but not written, its validity status is 
not changed. %Two cases are possible:

\noindent$(1)$ $R~\abs x \in \AM$. By 
a  case 
analysis on $\sigma(x)$ and $\sigma(\abs x)$ we have:
$\sigma_s(x)\!=\!(V,I)$ and  $\sigma_s(\abs x)\!=\!(V,I)$, or $\sigma_s(x)\!=\!(V,V)$ 
and  
$\sigma_s(\abs x)\!=\!(V,I)$ or $(V,V)$.  Reading has no effect on validity status and in 
all 
cases we have $\sigma'(\abs x)~\leq~\sigma'(x)=\sigma_s(x)$.

\noindent$(2)$ $RW~\abs x \in \AM$ then similarly to the case (2) above we have 
$\sigma_s(\abs x)=(V,I)$, additionally $\sigma'(x)=\sigma_s(x)=(V,I)$ or $(V,V)$. In all 
cases $\sigma'(\abs x)\leq\sigma'(x)$.

\noindent$\bullet$ If  $x$ is not accessed but is in the declaration, the 
reasoning is the same as if it was only read. 
Note that the variable cannot be declared 
in write mode, $W~\abs x \in \AM$, by
Definition~\ref{def-WD}.
\end{proof}

Finally, a well-declared program always runs to completion: it never tries to access an 
invalid memory location.

\begin{Theorem}[Progress for well-declared programs]\label{thm-progress}
If a program $\Prog$ is well-declared, then its execution cannot reach a stuck 
configuration.
\end{Theorem}

\begin{proof}
 By 
Property~\ref{prop:stuck}, 
it is sufficient to prove that 
unification on the validity status is always possible. 
We consider a reduction  $(\transl{\AM\{S\}},\sigma) \to^* (S,\sigma_s) \to^* \ldots$ 
similarly  to the proof above.

By definition of well-declared 
programs and because of the signature of effects ($w~x$ cannot be stuck), only two 
cases have to be analysed for the local operations (plus two similar cases for remote 
statements):
 \begin{itemize}
\item \Pull\ operations (on $x$ and $\abs x$) in the translation of $R~\abs x$ or 
$RW~\abs x$. Unification 
requires that $\sigma(x)=(X,V)$ and  $\sigma(\abs x)=(Y,V)$.
\item $r~x$ operation in the evaluation of $S$. Unification 
requires that $\sigma'(x)=(V,X)$ where $\sigma'$ is the store in which the read access is 
to be evaluated.
\end{itemize}
Indeed, access mode declarations do not generate reading operations, and by definition 
function statements contain no \Push\ or \Pull.

Concerning the first case, because of Theorem~\ref{thm-SR}, we have $\sigma(\abs x)\leq 
\sigma(x)$, and because of property~\ref{prop:safe} none of them is $(I,I)$. By case 
analysis on the possible values of $\sigma(\abs x)$ and $\sigma(x)$, it is easy to show 
that $\sigma(x)=(X,V)$ and  $\sigma(\abs x)=(Y,V)$ if we reach the two \Pull\ statements 
that perform data transfers before the execution of the function.

Concerning read access, they should be verified by an induction on the reduction steps 
following the state $(S,\sigma_s)$ showing that, for any variable $x$ that is declared $R$ 
or $RW$, in all states we have $\sigma'(x)=(V,X)$. Indeed, by the same analysis as in the 
proof of 
Theorem~\ref{thm-SR} we know that $\sigma'(x)=(V,X)$. Because of 
Property~\ref{prop-localised} no remote operation is possible on $x$ and thus only $r~x$ 
and $w~x$ operations are possible on $x$, both maintain the invariant $\sigma'(x)=(V,X)$ 
for some $X$.
%
%This is done by case analysis of the effect applied in the translated program, and 
%showing that in each case the effect rule can be applied.
%Take the example of \Pull. \Pull\ only appears upon $R~x$ or $RW~x$ and when  
%$\isvalid~x$ is 
%false. In this case $\sigma(x)=(I,V)$ by property~\ref{prop:safe}. Consequently \Pull\ 
%is 
%not stuck.
%Now considering the effect $r$, $r~x$ can only be obtained by translation of $R~x$. 
%Consequently, $r~x$ appears necessarily after a conditional \Pull, thus when we reach 
%$r~x$ we necessarily have $\sigma(x)=(V,I)$ or 
%$(V,V)$ (by a trivial case analysis of the possible validity status of $x$ before the 
%\Pull). 
%Other cases are similar.
\end{proof}

Considering the example above of a variable written on the CPU, and then read on the GPU, 
a well-declared program encoding this behaviour would be  $RW~\abs x\{w~x\};\rem{R~\abs 
x}\{\rem{r~x}\}$. This code automatically generates the \Push\ instruction that prevents 
the program from being stuck.
\subsection{Effects and Access Mode Declarations for Arrays}\label{sec-arrays}
In array structures, the 
validity status of the whole array is abstracted away by a single validity status pair. 
We extend the syntax for arrays as follows, $x[i]$ denotes the indexed access to an 
element of the array. More precisely the new operations on arrays and their elements are 
(we still have the previous operations on non-array and abstract variables):
\[S::= ... \,|\, r~x[i] \,|\, w~x[i] \]

Synchronisation operations (\Push\ and \Pull) exist for arrays but the whole array is 
synchronised, and we write $\Push~x$ and $\Pull~x$ as above.
All the elements of the array are represented by a single abstract variable: $\abs{x}$ 
represents a safe abstraction of the validity status of all $x[i]$. In other words, as soon as one element of the array $x$ is invalid locally (resp. remotely) the validity status of $\abs x$ can only be $(I,V)$ (res. $(V,I)$).

The semantics of access mode declarations and programs is unchanged because 
synchronisation operations and access mode declarations do not concern array elements.
The concept of well-protected programs must be adapted to the case of array structures, 
and more 
precisely to the fact that several  memory locations are represented by a single abstract 
variable.


\begin{definition}[Well-declared program with array access]\label{def:well-declared-array}
A program $\Prog$ is \emph{well-declared} if for all $\AM\{S\}$ in $\Prog$, additionally 
to the rules of Definition~\ref{def-WD}, we have:
\begin{itemize}
\item $w\ x[i]\in S \implies (W\ \abs x \in \AM \lor RW\ \abs x \in \AM)$,
\item $r\ x[i]\in S \implies (R\ \abs x \in \AM \lor RW\ \abs x \in \AM)$,
\item $W\ \abs x\!\in\! \AM \!\implies\! \forall i\!\!\in\!\!\range(x).\, w\ x[i]$ occurs in all execution 
paths of $S$,
\item Plus the same rules for remote operations.
\end{itemize}
Where $\range(x)$ is the set of valid 
 indexes for an array $x$.
\end{definition}

The other properties are expressed similarly and both \emph{subject-reduction}, 
Theorem~\ref{thm-SR}, and \emph{progress}, Theorem~\ref{thm-progress}, are 
still valid. The only change is the ``correct abstraction of the memory state'' criteria 
that becomes $\forall x\in 
\vars(S).\,\forall i\in\range(x).\, \sigma(\abs x)\leq\sigma(x[i])$ instead of $\forall 
i\in\range(x)$ for arrays. The proofs are similar except in the case of $W~x$ 
declarations where the fact that all elements of the array must be written is necessary 
to ensure that no element is in the status $(I,V)$ (which could not be safely represented 
by $(V,I)$) at the end of the function execution. If we focus on the proof of 
Theorem~\ref{thm-SR}, case ``$x$ is written, sub-case (1) we have $\sigma'(\abs x)=(V,I)$ 
which is a safe abstraction because \emph{all elements have been written}, and thus 
$\sigma'(x[i])=(V,I)$ for all $i$. If one element $j$ was not written, we could have had 
$\sigma'(x[i])=(I,V)$ which would invalidate the theorem. Overall, only arguments about correctness of the abstractions need to be adapted, and we can prove the following theorem.
%
%
%\subsection{Towards Array Accesses: Adding an abstraction layer}
%In the simple formalisation provided here, we only consider simple memory locations and 
%consequently there is no need to have an abstract representation of a set of memory 
%locations. We can consider that the manipulation of the VectorPU abstract structure 
%representing the memory and the memory itself as the same entity. While this point of 
%view allowed us to provide a simple and precise enough representation, and to prove the 
%correctness of the general approach, it becomes not close enough to the real 
%implementation if we want to reason on array structures. In array structures, the 
%validity status of the array is abstracted away by a single validity status pair. Then 
%approximations inspired from classical abstract interpretation techniques~\cite{Cous77} 
%can be used except that the abstract view is computed dynamically but must stay 
%consistent with the validity status of the real memory.
\begin{Theorem}\label{thm-correct-array}
If a program using arrays is well-declared according to Definition~\ref{def:well-declared-array} then its execution verifies both the subject-reduction and the progress property.
\end{Theorem}
\subsection{Discussion: Similarities and differences relatively to VectorPU}



Let us compare the formal definition of the coherence protocol, Figure~\ref{sem-AM} (valid for simple memory locations or arrays), with the VectorPU implementation of the protocol for simple arrays, Figure~\ref{fig:vectorpucoherence}. Except from the order of operations and minor changes, the structure is the same and the form of operations is similar. The main difference is that there is no view of abstract vs. concrete variables, however, if we consider that transfer operations on abstract variables have no effect, and that validity status of concrete variables can be abstracted away, the code is  the same as the formalisation. 

Taking a more global point of view, no verification is performed by the VectorPU framework and thus, the property of ``well-declared programs'' is not checked currently by the framework. Such a check could either be done by a static analysis (involving some approximations, meaning  some correct programs could be rejected), or by runtime verification checking that each function performs exactly the required access. The first solution would require more development and might be tricky, while the second one is not acceptable considering the target application domain because of the overhead involved by the dynamic checks. In the current state of the library, the property ``well-declared programs''  must be ensured by the programmer. The implementation of VectorPU relies on the hypothesis that the function declarations are correct, because of this the current formalisation is a significant step forward as it allows us to express precisely what assumption is made by the library on the programmer's code.

Finally concerning expressiveness, the ``well-declared programs'' definition is a bit more flexible than what VectorPU targets at the moment because it is safe to declare in our framework functions that access some variables on the CPU and others on the GPU, and this is not planned in VectorPU, again due to the supported usage scenario.

These differences highlight interesting improvement directions for VectorPU while we can still consider that the current article  is a faithful formalisation of the library. In the next section, we investigate a feature that is not yet supported by VectorPU, but exists in SkePU. We can somehow consider the theoretical results below as a specification of a future extension of the library.

\section{Overlapping arrays}\label{sec:overlap-array}

In the preceding section we supposed that arrays were well-separated. The preceding abstraction would also be valid for an array that would be split into disjoint entities (such as
VectorPU \texttt{pvector}s) and always used either as the disjoint sub-arrays or as the whole. In this section we extend the framework to take into account overlapping arrays. Here we still consider single dimension arrays for simplicity but multi-dimensional arrays could easily be taken into account.

\subsection{Context and Objectives}

In VectorPU, the first \texttt{pvector} on a vector passed as argument for access on device will (over)allocate space for the entire \texttt{vector}, all subsequent \texttt{pvector} accesses to the same \texttt{vector} can skip the allocation.
 Consequently, two successive data transfers of the same memory location will be written to the same memory location, even if the two initial locations are accessed through different (overlapping) \texttt{pvector}s.
Consequently, on the formal side, if several push or pull are performed on overlapping memory locations, the transferred memory has the same overlaps as the source. This is very important to ensure that no two copies of the same array element will coexist in the same location.

\paragraph{Problem statement}
Overlapping arrays raises several difficulties making the approach currently adopted in VectorPU not adapted. Indeed a single write operation can change the validity status of a cell that belongs to several arrays. Consequently, several access annotations may have to be written for a single operation. As a consequence, some annotations may never be correct: a function that is declared to write all the elements of an array $x$ will necessarily write some elements of the arrays that have an overlap with $x$, these overlapping arrays should thus be annotated and transmitted, or another coherence protocol should be used.

\paragraph{Approach}
In this work we take the decision not to change the coherence protocol of VectorPU and instead work at the access mode declaration level to ensure the consistency of overlapping arrays.

To take into account overlapping arrays in VectorPU, two approaches could be envisioned. A naive solution consists in applying the results for non-overlapping arrays. Indeed, Definition~\ref{def:well-declared-array} of well-declared programs with array accesses is still valid. However,  the programmer now has to annotate more variables because each array access operation may involve several arrays. Due to the array overlap, the programmer should now know all the arrays that are impacted by a function execution, including the overlapping arrays that are not passed as parameter, and the library should be extended to pass them as ``artificial'' parameters. To be more explicit, Definition~\ref{def:well-declared-array} should be extended as follows (with a symmetrical rule for remote writing):

\begin{quote}
For every $y$ overlapping $x$, if $i$ is in the range in common between $x$ and $y$, we have $w\ x[i]\in S \implies (W\ \abs y \in \AM \lor RW\ \abs y \in \AM)$.
\end{quote}

First note that no additional rule is necessary for read operations. Indeed, read operations use the validity status but do not modify it, consequently, read operations do not modify the validity status of other arrays and have no consequence on overlapping arrays.
Note also that the above rule restricts a bit the expressible effects as, for example. $W\ \abs x ; \rem{R}\ \abs y$ cannot be valid if $x$ and $y$ overlap.


\medskip

Though very precise, this approach does not seem realistic and we additionally develop an inference mechanism for access mode declarations in presence of overlapping arrays.
The objective  is to infer the correct annotations on  variables that are not passed as parameters. Knowing the access modes for the function parameters, we infer what operation must be done on other intersecting arrays to ensure the coherency of the system, and we express these additional operations as implicit generated annotations. These additional access mode declarations are inferred, the semantics of these added declarations will result in additional data transfers and validation/invalidation operations that make the program correct. This is less precise as it takes a pessimistic approach on the operations performed by the declared arrays. For example, for any array that is declared $RW$ we will suppose that all the array elements may be read and written, but the approach is safe and mostly automatic. This is presented in Section~\ref{sec:infer-overlap} below.
%
%Another way to see the next two subsection is that the first one extends the well-declared definition for overlapping arrays, and the second one defines an inference mechanism that ensures that a function is well-declared in presence of overlapping arrays, provided the declaration is correct not taking overlapping arrays into consideration, i.e. in the sense of Definition~\ref{def:well-declared-array}.



\subsection{Access mode inference for overlapping arrays}\label{sec:infer-overlap}
Starting from a given set of access mode annotations, we want to infer other access modes that are consequences of the overlaps and the existing annotations.
Because we are only aware of an approximation of the effects (for each array variable, effect is abstracted by a single global effect), the inferred accesses will be approximate but can be a safe over-approximation of the effect of the function.
 Without knowing the real accesses performed by the function, we deduce from the declared access modes, a set of additional ``artificial accesses''.

We consider $\AM$ the set of all access modes declared for a given function and extend it so that the function satisfies the well-declared program requirement even with overlapping arrays.

\begin{table}[!tb]
\begin{mathpar}
\inferrule
{E\ \abs x \in\AM}
{E\ \abs x \in\Overlap \AM}

\inferrule
{\rem{E\ \abs x} \in\AM}
{\rem{E\ \abs x} \in\Overlap \AM}


\inferrule
{W\ \abs x \in\AM \\ W\ \abs y \not\in\AM \\ x \text{ and } y \text{ overlap}}
{RW\  \abs y \in\Overlap \AM}

\inferrule
{RW\ \abs x \in\AM \\ x \text{ and } y \text{ overlap}}
{RW\  \abs y \in\Overlap \AM}

\inferrule
{\rem{W\ \abs x} \in\AM \\ \rem{W\ \abs y} \not\in\AM \\ x \text{ and } y \text{ overlap}}
{\rem{RW\  \abs y} \in\Overlap \AM}

\inferrule
{\rem{RW\ \abs x} \in\AM \\ x \text{ and } y \text{ overlap}}
{\rem{RW\  \abs y} \in\Overlap \AM}

\end{mathpar}
\caption{Extension of access mode annotations to deal with overlapping arrays}\label{Overlap-ext-tab}
\end{table}
To understand the principle of the approach, consider the case where 
$W\ \abs x\in \AM$ then $ \forall i\in\range(x). w\ x[i]$ occurs in all execution 
paths of $S$, and thus for all arrays $y$ overlapping $x$  we must have $(W\ \abs y \in \AM \lor RW\ \abs y \in \AM)$. If we have no additional information we will ensure that $RW\ \abs y$ is also in the set of access mode declarations, which is always safe.


\begin{definition}[Extension of access mode declarations for overlapping arrays]\label{def-overlap-annotation}
Consider a set $\AM$ of access mode declarations.
 The extension for overlapping arrays of $\AM$  is the smallest set $\Overlap \AM$ defined by the  rules in Table~\ref{Overlap-ext-tab}.


\end{definition}

The following theorem states that if the set of function parameters is extended according to the preceding extension, then memory consistency is ensured. Note that it means that a set of artificial parameters are to be added to some functions, in the sense that  data-transfers and validity status modifications  have to be performed on vectors that are not among the original parameters of the function.

\begin{Theorem}\label{thm-correct-array-overlap}
Consider a program that is well-declared according to Definition~\ref{def:well-declared-array}, not taking into account  overlapping arrays. Suppose its access mode declarations are extended according to Definition~\ref{def-overlap-annotation} then the execution of the obtained program verifies both subject reduction and progress, even in presence of overlapping arrays.
\end{Theorem}

\begin{proof}[Proof sketch]
The principle of the proof is to prove that, provided a set $\AM$ correctly declares the accesses performed syntactically by a function on its parameters, the set $\Overlap \AM$ is a correct approximation of the accesses performed by the function on all the arrays of the program, i.e. the parameter arrays and the arrays that overlap the parameter arrays. Then Theorem~\ref{thm-correct-array} will be  sufficient to conclude.

Trivially, the first two rules of Table~\ref{Overlap-ext-tab} are sufficient to conclude about normal function parameters. We now need to ensure that operations on overlapping arrays are well-declared. We focus on non-remote operations and prove that (indirect) operations on overlapping arrays are well-declared, according to Definition~\ref{def:well-declared-array} modified by the additional rule introduced in above: 
\begin{quote}
For every $y$ overlapping $x$, if $i$ is in the range in common between $x$ and $y$, we have $w\ x[i]\in S \implies (W\ \abs y \in \AM \lor RW\ \abs y \in \AM)$.
\end{quote}
By a simple case analysis on the possible annotations and the possible operations performed on the arrays, we deduce that the access modes added by the four last rules of Table~\ref{Overlap-ext-tab} are sufficient to ensure this rule and the symmetrical one for remote writing.
\end{proof}



\subsection{Towards an implementation in VectorPU}


% Tell what is needed
To implement the proposed mechanism that ensure the safety of overlapping vector accesses
in a function call $f$, we need to add to VectorPU the two following components:
\begin{itemize}
    \item A representation $r_v$ that allows to retrieve, for any given \texttt{pvector} $pv$
             of a \texttt{vector} $v$, the set of all other valid \texttt{pvector}s of $v$ 
             that overlap with $pv$.
             $r_v$ is initialized as empty when declaring a new \texttt{vector} $v$,
             queried and/or updated at \texttt{pvector} creations, deletions, and at calls,
             and is removed when $v$ is deallocated.
          %At the vector declaration phase of a program,
          %the representation is constructed by overloading the vector constructor.
          %When a vector finishes its life cycle,
          %it is erased from the representation.
    \item A mechanism which intercepts the function call $f$ and,
          for every vector operand (\texttt{vector} or \texttt{pvector}) $pv$ 
          accessed as W or RW in $f$,
          looks up in the corresponding representation $r_v$ all  
          \texttt{pvector}s overlapping with $pv$. %for every vector $v$ in the argument list
          %of the function call $f$. Then we perform the conservative coherence actions on those vectors\TODO{This should refer back to the technique in Theorem 4, i.e., automatically rewrite the given access mode R or W of an overlapping new operand pvector into RW, to be safe.} retrieved by the look-up.
          For other arguments in $f$ that overlap with $pv$ and have access mode
          R or W, their access mode
          is updated to RW, as proposed in Table~\ref{Overlap-ext-tab}.
          For any other existing \texttt{pvector}s $w$ of $v$ not accessed in $f$
          (but possibly in earlier and/or later calls) 
          that do overlap with $pv$, 
          we append shadow arguments $RW(w)$ to $f$ as suggested by Table~\ref{Overlap-ext-tab}.
          Finally, the intercept mechanism performs, as before, the resulting coherence
          actions (data transfers, status updates) and delivers the call.
          For intercepting the function call, the function call operator is overloaded.
          %\TODO{Can such overloading be done automatically, without programmer assistance? - C.}
\end{itemize}

As a simple example, let us consider the following set of \texttt{pvector}s and call sequence:

\begin{verbatim}
v = new vector(10, ...);
pv1 = new pvector( v, [2:5] );
pv2 = new pvector( v, [4:8] );
pv3 = new pvector( v, [7:9] );
pv4 = new pvector( v, [2:3] );
...
f1( ... R(pv1), R(pv2), ... );
f2( ... W(pv3) ... );  
f3( ... RW(pv4), R(pv2), ... );
\end{verbatim}

Intercepting the function calls, we maintain $r_v$ and update the calls as 
described above.
For the call to \verb.f2., we infer from W(pv3) and the overlap of \verb.pv3. with \verb.pv2.
by Table~\ref{Overlap-ext-tab} that the access mode of \verb.pv2. (not accessed in \verb.f2.) must be 
upgraded to RW, which we do by conceptually appending \verb.RW(pv2).
as a shadow argument to \verb.f2.. The call to \verb.f2. is thus
conceptually rewritten\footnote{As all overlapping
\texttt{pvectors} had been identified before the rewriting, the rule needs
not be applied recursively to the appended shadow arguments, here \texttt{RW(pv2)}.} into

\begin{verbatim}
f2( ... W(pv3) ... , RW(pv2) ); 
\end{verbatim}

\noindent
hence we make sure that the access to \verb.pv2. in the subsequent
call to \verb.f3. will be handled correctly. 
%This leads to invalidation of \verb.pv2., i.e., removal from $r_v$,
%before delivering the actual call to \verb.f2..

%Analogously, we infer at call \verb.f3. that \verb.pv1. must be
%invalidated due to its overlap with argument \verb.pv4..
 
% Summarise the requirement for data structure
%For the first component, we can see that a graph representation is needed
%\TODO{Why a graph? A simple ordered list (or interval tree, for optimization) of valid device copies (pvectors) should suffice? SkePU uses an ordinary list (C++ STL map) for this purpose.}.
%Given a simple three-vector set, one can easily construct a scenario
%that all three vectors overlap with each other in memory address ranges.\TODO{Hard to follow. %If it is important, could you please show some example scenario?}
%For the second component, we notice that the insertion, deletion and the look-up operators is required.
%To summarise, we need a data structure that can represent a graph structure and support efficient insertion, deletion
%and the look-up operators.

% given the requirement, select a data structure that fulfil the requirement
%We can choose dense 2-dimensional matrix to represent the graph.\TODO{Lu, please check if a graph is really necessary here. Also, I do not see how a fixed-size dense matrix could be used here, as valid copies are added and removed dynamically as GPU calls are executed. Or do I overlook something? - C.}
%We need to know the size of 2-dimensional matrix in advance.
%We assume the number of vectors in a program is reasonably small
%so that a dense matrix will not take noticeably large amount of memory
%and we can set in advance a reasonable number for the size,
%or rely on the static analysis of a compiler to estimate a conservative size value.
%The insertion of an overlapping array pair requires to write a 1 in a cell of the dense matrix,
%thus only takes $O(1)$ in time, the deletion operator requires to write a 0 in a cell
%thus also takes $O(1)$. The look-up operator only requires to return a row of the matrix,
%thus also takes $O(1)$. Since we only need a binary number in a cell of the matrix,
%thus the memory size of the dense matrix representation can be further reduced.
%To summarise, the 2D dense matrix representation allows to represent the graph
%and support the required operators efficiently.

It remains to select an appropriate data structure for $r_v$ that allows for
efficient dynamic insertion and removal of \texttt{pvector}s of a vector $v$, 
i.e., index intervals, and efficient lookup of all pvectors that overlap
with a given query interval.
For very small numbers of \texttt{pvector}s of a vector $v$, a simple unordered
list of \texttt{pvector}s is sufficient; this is used e.g.\ in the
smart-container coherence management in SkePU \cite{Dastgeer-IJPP15}. 
For scaling up to larger numbers of \texttt{pvector}s,
a \emph{segment tree} \cite[Sec.~10.3]{Overmars} % : Segment Trees, pp.~224–227
could be used. A segment tree storing $n$ intervals
can be updated dynamically (insertion, removal) in time  $O(\log n)$ and
can retrieve the set of all $k$ intervals overlapping with a query interval
in time $O(k+\log n)$; the space requirements is $O(n\log n)$.

% 180916 CK: I disabled the SkePU section for now,
% as we now have the better VectorPU pvector use case instead.
% For the comparison to SkePU we can write something shorter,
% if at all (just explain why it is more difficult,
% see also the corresponding added remarks in the Conclusion).
%
%\section{Case study for overlapping array accesses: SkePU}

Skeletons are pre-defined generic program building blocks,
inspired by higher-order functions,
that implement certain frequently occurring patterns of
computation and dependences, for which efficient parallel
or platform-specific implementations might be available.
Skeletons can be parameterized in problem-specific'
(usually, sequential) code to generate executable functions
with parallel implementations that can be called like
any handwritten implementation.

SkePU \cite{Enmyren10,Ernstsson18} 
is a C++ based skeleton programming framework for heterogeneous systems.
It provides a number of general-purpose data-parallel skeletons
(generalized map, reduce, mapreduce, stencil, scan), each of
which has implementations (backends) for execution on one and multiple CPU cores, as
well as CUDA and OpenCL backends for execution on single and multiple GPUs.
While it is possible to enforce using a specific backend (and thus,
execution platform) for a skeleton call or even an entire program
run, SkePU provides autotuned backend selection, which automatically 
decides at runtime,
based on internal performance models for the current target system, 
where to best execute a 
skeleton call in order to optimize for time or energy. 

All (non-scalar) operands passed into or out of skeleton calls must be
SkePU data containers, which wrap C++ datatypes;
currently, extensions of STL \texttt{vector<...>}
and a \texttt{matrix} data type are provided.
Using iterators, SkePU skeleton calls can work on dense sub-arrays
defined by arbitrary index intervals.
The SkePU data containers, improved in their second generation for
CUDA devices \cite{Dastgeer-IJPP15}, intercept all single-element
or bulk accesses to the encapsulated C++ arrays and
internally perform coherent software caching of
accessed element subsets in non-shared device memories, 
using lazy data copying to avoid unnecessary data moves.
They also transparently optimize the memory management
by lazy deallocation of invalidated device copies of element
subsets and reusing such space of matching size for new copies.
A smart copying technique locates, for each (sub)array access, the 
subranges with valid copies nearest to the accessing device
and thereby optimizes data transfer times additionally,
also using direct device-to-device copying where applicable.


\vspace{1.4mm}
\noindent
{\em Implementation notes }
SkePU\footnote{SkePU source code and documentation: http://www.ida.liu.se/labs/pelab/skepu} 
is organized as a C++ source-to-source precompiler based on 
LLVM clang and an include library.
The part of the SkePU source code dealing with coherence in SkePU 
is very large, over 1500 lines of C++ code in total that are 
spread over several source files. Hence, 
the coherence-relevant code of SkePU is much less
localized, and it is also more complex than that of VectorPU
because SkePU works with a more complex coherence protocol
and the device type and number is not hardcoded but can be 
arbitrarily large (VectorPU implicitly assumes 1 CPU and 1 GPU).
% Some code occurs many times, e.g. in different
% access operators or for different container initializers.
%
% Part of the coherence code for overlapping array accesses
% is not nicely written (lack of comments, some warning comments
% about unfixed or untested corner cases)
% but otherwise seems to make sense to me.

Each SkePU data container stores all its elements in main memory,
referred to as the \emph{main copy} and
(part of) elements accessed in skeleton calls might also reside
in one or separate device memories, referred to as \emph{device copies}.
The container itself resides in main memory as all decisions
about operand data movement etc.\ before and after a skeleton call 
are made on CPU, regardless where the skeleton call will execute, 
hence container metadata such as the main copy coherence state 
and those of existing device copies are stored in main memory (only).

The coherence state of the main copy (in main memory) 
is, basically, represented by 2 boolean flags: 
\verb+m_valid+ (initialized to true in container constructors) and 
\verb+_noValidDeviceCopy+ (initalized to true), as described below.

Device copies are identified by pairs consisting of the
address of the first element accessed and the number of
elements accessed. 
Multiple device copies of the same array can overlap 
even on same device (if they are readonly)
and multiple copies can exist even with write access if their element ranges do not overlap.
%
Different from the main copy, the current coherence state of device copies 
is \emph{implicitly} modeled by their membership in lists
that are maintained in the container in main memory at runtime: 
\begin{itemize}
\item list of valid device copies on each device
\item list of modified (valid) device copies on each device with main copy not updated yet. See also Figure~\ref{fig:skepucoherence1}.
\end{itemize}

\begin{figure}
\begin{small}
\begin{verbatim}
template <typename T>
class Vector
{   ...
 private:
  T *m_data; // array of data elements
  mutable bool m_valid; // keep track if main copy is valid or not
  size_type m_size;
  mutable bool m_noValidDeviceCopy;
  ...
  mutable std::map<std::pair<T*, size_type>, device_pointer_type_cu>
                             m_deviceMemPointers_CU[MAX_GPU_DEVICES];
  // list of copies changed on device but not synced with host memory:
  mutable std::map<std::pair<T*, size_type>, device_pointer_type_cu>
                    m_deviceMemPointers_Modified_CU[MAX_GPU_DEVICES];
  ...
}
\end{verbatim}
%   size_type m_capacity;
%   bool m_deallocEnabled;
\end{small}

\vspace{-3mm}
\caption{\label{fig:skepucoherence1}The main data structures managing
  coherence in the SkePU \texttt{Vector} container implementation (file 
   \texttt{vector.hpp}).}
\end{figure}

The coherence protocol used is a variant of MSI, with 3 states for the main copy:
\begin{itemize}
\item M: exclusive owning for reading and writing by CPU.\\
         Flags \verb+m_valid==true+ and \verb+m_noValidDeviceCopy==true+.
\item S: shared for multiple readers.\\
         Flags \verb+m_valid==true+ and 
               \verb+m_noValidDeviceCopy==false+.
\item I: invalid in main memory.  Flag \verb+m_valid==false+.
\end{itemize}

Interestingly, it is not explicitly represented if only \emph{part} of 
the main copy is invalid while another part is still valid. 
This is implicit, by having the still valid parts not
being members of the list of modified device copies.
Read and write accesses to single container elements 
are internally distinguished between
by using a container proxy class.

% Code snippets: it should maybe not exceed one page.
% Some manual "inlining" and simplification of the source code
% will be necessary to make it readable.

Due to space limitations we can here only display a few selected 
snippets of the SkePU container coherence code to illustrate the mechanism.
Figures \ref{fig:skepucoherence1}--\ref{fig:skepucoherence3} show small
samples of (simplified) coherence-related
code from files \texttt{vector.hpp}, %\texttt{vector/vector.inl} 
 \texttt{vector/vector\_cu.inl} and \texttt{backend/device\_mem\_pointer\_cu}.

  
%\begin{figure}
%\begin{small}
%\begin{verbatim}
%// function updateDevice_CU manages container state for
%// read and write accesses on a CUDA device:
%
%if(m_noValidDeviceCopy)  m_noValidDeviceCopy = false;
%   
%typename std::map<std::pair< T*, size_type>, device_pointer_type_cu >::iterator result;
%std::pair< T*, size_type > key(start, numElements);
%typename std::map<std::pair< T*,size_type >, device_pointer_type_cu >::const_iterator %it;
%result = m_deviceMemPointers_CU[deviceID].find(key);
%device_pointer_type_cu tempCopy = NULL;
%if (result == m_deviceMemPointers_CU[deviceID].end()) {
%  //insert new, alloc mem and copy
%  tempCopy = ...
%  if (hasReadAccess(accessMode))
%     copyDataToAnInvalidDeviceCopy(tempCopy, deviceID, streamID);
%  result = m_deviceMemPointers_CU[deviceID].insert(
%    m_deviceMemPointers_CU[deviceID].begin(), std::make_pair(key,tempCopy));
%}
%else { //already exists but need to update contents:
%   tempCopy = result->second;
%   if (hasReadAccess(accessMode) && tempCopy->isCopyValid()==false)
%     // check whether the copy is invalid and we need to copy data:
%         copyDataToAnInvalidDeviceCopy(tempCopy, deviceID, streamID);
%}
%// Before returning, mark all other copies as invalid
%// if writing this copy and they are overlapping with this copy */
%if (hasWriteAccess(accessMode)) {
%  // add this copy to modified list:
%  m_deviceMemPointers_Modified_CU[deviceID].insert(
%    m_deviceMemPointers_Modified_CU[deviceID].begin(), std::make_pair(key,tempCopy));
%  m_valid = false;
%  // mark only local copies invalid. Each device will do that in
%  // multi-gpu execution for each single call. ...
%  if (markOnlyLocalCopiesInvalid) {  ... }
%  else { // Mark all overlapping copies from all devices as invalid:
%    for (int devID = 0; devID < MAX_GPU_DEVICES; ++devID) {
%      if (m_deviceMemPointers_CU[devID].empty()) // no copies...
%         continue;
%      for (it = m_deviceMemPointers_CU[devID].begin();
%           it != m_deviceMemPointers_CU[devID].end(); ++it) {
%        if (tempCopy != it->second && it->second->isCopyValid()
%            && tempCopy->doCopiesOverlap(it->second, true)) {
%          // this is possible considering gpu-gpu transfers and
%          // in some other cases e.g. map(v1 RW); ... map2(..., v1 W);
%          if (it->second->deviceDataHasChanged()) {
%   //                      assert(false); /*! TODO: fix this */
%                     /*!
%                     * if not fully overlapped then need to transfer as some data % should be written back to device memory
%                     * if fully overlapped then no need to update it as it is %overwritten in current copy...
%                     */
%                     if(it->second->doOverlapAndCoverFully(tempCopy) == false)
%                     {
%                        it->second->copyDeviceToHost();
%                     }
%
%                     /*! should delete this copy from this list as it needs not to be %updated back... */
%                     assert(m_deviceMemPointers_Modified_CU[devID].find(it->first) != %m_deviceMemPointers_Modified_CU[devID].end());
%                     m_deviceMemPointers_Modified_CU[devID].erase(m_deviceMemPointers_M%odified_CU[devID].find(it->first));
%                  }
%
%                  /*! mark copy invalid */
%                  it->second->markCopyInvalid();
%               }
%            }
%         }
%      }
%   }
%\end{verbatim}
%\end{small}

%\vspace*{-3mm}
%\caption{\label{fig:skepucoherence2}Selected coherence code from the SkePU vector container implementation, here for write accesses on a CUDA device, 
%in file \texttt{vector\_cu.inl}.}
%\end{figure}
  

% \begin{figure}
%\begin{small}
%\begin{verbatim}
%// from vector_proxy.inl:
%...
%//This is where values are being read:
%template <typename T>
%Vector<T>::proxy_elem::operator T&() const
%{
%   update(m_parent);
%   return m_parent.m_data[m_index];
%}
%
%//This is where values are being written:
%template <typename T>
%typename Vector<T>::proxy_elem& Vector<T>::proxy_elem::operator=(const proxy_elem& rhs)
%{
%   update(rhs.m_parent);
%   updateAndInvalidate(m_parent);
%   m_parent.m_data[m_index] = rhs.m_parent.m_data[rhs.m_index];
%   return *this;
%}
%\end{verbatim}
%\end{small}
%}
%
%\vspace*{-3mm}
%\caption{\label{fig:skepucoherence2a}Code excerpt from the SkePU proxy
% vector for single-element accesses of a vector (parent) .}
%\end{figure}
  
\begin{figure}
\begin{small}
\begin{verbatim}
// Read on CPU via [] operator:
template <typename T>
const T& Vector<T>::operator[](const size_type index) const
{
  // updateHost() inlined, calls updateHostCU() and sets m_valid:
  if (deviceID < 0) { // do it for all devices....
    for (deviceID = 0; deviceID < MAX_GPU_DEVICES; ++deviceID) {
      if (m_deviceMemPointers_Modified_CU[deviceID].empty())
        continue;
      for (it = m_deviceMemPointers_Modified_CU[deviceID].begin();
           it!=m_deviceMemPointers_Modified_CU[deviceID].end(); ++it)
        // At one point in time, there could be >1 valid
        // "modified" copy per each device for a container */
        it->second->copyDeviceToHost();
      m_deviceMemPointers_Modified_CU[deviceID].clear();
      ...
    }
  }
  else if (!m_deviceMemPointers_Modified_CU[deviceID].empty()) 
    ... similar, do it only for given deviceID
  m_valid = true;
  return m_data[index];
}
\end{verbatim}
\end{small}

\vspace*{-3mm}
\caption{\label{fig:skepucoherence2b}Coherence code excerpt from \texttt{vector.inl}
          handling certain element read accesses to a SkePU vector on CPU.}
\end{figure}
  
  
\begin{figure}
% // from device_mem_pointer_cu.h:
%
%  // Checks whether the copy passed as argument has a subset of elements range
%  // to the one that object points to:
%  template <typename T>
%  bool DeviceMemPointer_CU<T>::doOverlapAndCoverFully (
%                       DeviceMemPointer_CU<T> *otherCopy )
%  { if (m_hostDataPointer <= otherCopy->m_hostDataPointer
%         && (m_hostDataPointer + m_numElements)
%             >= (otherCopy->m_hostDataPointer + otherCopy->m_numElements) )
%        return true;
%      return false;
% }
%
%    // returns true if there exist any range (that needs to be written) that
%   // is overlapping to the otherCopy:
%   template <typename T>
%   bool DeviceMemPointer_CU<T>::doCopiesOverlap ( DeviceMemPointer_CU<T> *otherCopy,
%                                                  bool oneUnitCheck)
%   {
%     if (oneUnitCheck)
%        assert(m_numOfRanges == 1 && otherCopy->m_numOfRanges == 1);
%     if ...
%
\begin{small}
\begin{verbatim}
// creating new device copy: allocate space in device memory
// and store a pointer to some data in host memory.
  ...
  /*! ranges that should be checked for overlap with other copies */
  m_rangesToCompare[m_numOfRanges++] =
                   std::make_pair( m_hostDataPointer, m_numElements );
  m_deviceDataHasChanged = false;
  ...
...
// returns true if there exists any range (that needs to be written)
// that is overlapping to the otherCopy:
template <typename T>
bool DeviceMemPointer_CU<T>::doCopiesOverlap(
                                DeviceMemPointer_CU<T> *otherCopy,...)
{  ...
  if (m_numOfRanges < 1)
    return false;
  for(size_t i=0; i<m_numOfRanges; ++i) {
    T *hostDataPtr = m_rangesToCompare[i].first;
    size_t numElements = m_rangesToCompare[i].second;
    if (hostDataPtr >= otherCopy->m_hostDataPtr &&
        hostDataPtr < otherCopy->m_hostDataPtr+otherCopy->m_numElements)
      return true;
    if (otherCopy->m_hostDataPtr >= hostDataPtr &&
        otherCopy->m_hostDataPtr < hostDataPtr + numElements )
      return true;
   }
   return false;
 }
 // ... for further overlap test functions see device_mem_pointer_cu.h
\end{verbatim}
\end{small}
%
%   // Checks whether there exists some overlap between elements range covered by current copy to the one passed as argument.
%   template <typename T>
%   bool DeviceMemPointer_CU<T>::doRangeOverlap( T *hostDataPointer, size_t numElements)
%   {
%     if (m_hostDataPointer + m_numElements <= hostDataPointer ) return false;
%     if (hostDataPointer + numElements <= m_hostDataPointer ) return false;
%     return true;
%   }
%   ...
\caption{\label{fig:skepucoherence3}Selected coherence code from the SkePU vector container implementation, in file \texttt{device\_mem\_pointer\_cu.h}.}
\end{figure}




% -----------------------------------------------------------
\section{A few related works}\label{sec:RW} 
Most of the verification works related to memory consistency focus on coherence 
protocols 
and/or 
weak memory models~\cite{pong1997verification}. 
Among them, one could cite~\cite{Gerth1999}, a formal specification of a caching 
algorithm, and its verification in TLA~\cite{Ladkin1999}. These works shows the 
difficulty to reason on memory coherency, but also that specifications in these models 
should rely on a few simple instructions on the type of memory accessed, a bit similarly 
to this proposal.
Coherence protocols have also been verified using CCS specifications~\cite{Barrio01}. 
These various works are quite different from the approach presented in this paper because 
we rely here on a declarative approach for memory accesses: the programmer declares the 
kind of memory accesses performed by a component, and the consistency mechanism ensures 
that each component accesses a valid memory space.

More recently, and adopting a more language oriented approach, Crary and 
Sullivan~\cite{Crary:POPL:2015} designed 
a calculus for expressing ordering of memory accesses in weak memory models, however we 
are interested here in a much simpler problem where memory access is  sequential 
and clearly identified. 
Even an  extension of this work for parallel processes would  result in a 
simpler model than the ones that exist for weak memory models because of the explicit 
consistency points introduced in the execution by the start/end of each function. 

The closest work to ours is probably~\cite{BJPTSAC16} that defines a memory access calculus similar to ours and prove the 
correctness of a generic cache coherence protocol expressed as part of the semantics of 
the calculus. Compared to this work, we are interested in explicit statements on memory 
accesses and thus the cache consistency is partially ensured  by the programmer 
annotations, making the approach and the properties proven significantly different. Some 
aspects of the approaches could however been made more similar, e.g. by extending our 
work to more than two address spaces or adopting a different  syntax. However our problem 
and formalisation are quite simpler, and we 
believe easier to read, while sufficient for our study.
The same authors also designed a formal model written in Maude~\cite{BJPTMaude16} to 
better understand the possible 
optimisations and the impact of the memory organisation on performance in the context of 
cache coherent multicore architectures. This could be an interesting starting point for 
future works, especially if we 
extend our work to better model the performance aspects of VectorPU and want to reason 
formally on the improved performance obtained by the library. Also from the same authors~\cite{Bijo2017} extends the results described above with parallel spawned task and could be a source of inspiration  to extend our work towards parallel function execution.

\section{Conclusion and future works}\label{sec:conclusion}
In this article we provided a formal approach to verify the consistency of the memory 
accesses in heterogeneous computer systems made of two memory spaces. We formalise the 
operations of memory accesses and memory synchronisation between the two memory spaces 
and prove that a program adequately annotated with informations on the memory accesses 
always access valid memory spaces and tracks correctly which of the memory space contains 
the up-to-date data.

The practical result is that we can verify the coherency mechanism used by the VectorPU 
library and ensure that, additionally to the significant performance benefits of the 
approach, the VectorPU mechanisms is correct and ensures the consistency of the memory 
accesses.

We also extended our model for studying the effect of
operations made on overlapping arrays.
The current implementation of VectorPU supposes that the 
(\texttt{pvector}) array operands always
represent disjoint memory locations, it does not 
take into account overlapping arrays. 
Based on the solution developed in our model,
we described an extension of the VectorPU library that could 
deal safely with overlapping array accesses by
overlapping \texttt{pvector} arguments.

We envision several extensions to this work.
% This old text from the conference was replaced
% by the previous new paragraph:
% ,  the most promising is the study 
% of  the operations made on overlapping arrays. 
% The current implementation of VectorPU supposes that the annotated memory 
% accesses deal with disjoint memory locations, it does not 
% take into account overlapping arrays. Designing an extension of the library that could 
% deal safely with overlapping array is one of the future direction we would like to 
% pursue. 
%Additionally, 
The current article only deals with two memory spaces; the extension to many 
memory spaces (as supported e.g.\ in SkePU)
seems relatively simple but  the mechanism dealing 
with memory transfers between several memory locations becomes a bit more complex; its 
formalisation should be similar.

Moreover, we are interested in the application of our approach to the 
verification of other frameworks. 
Indeed VectorPU uses the most primitive
cache coherence protocol, the VI-protocol.
 More elaborated coherence protocols like MSI or MESI 
 (as used e.g.\ in SkePU \cite{Dastgeer-IJPP15}) 
 introduce additional states where the
 number of readers has to be tracked for example. 
 Also, SkePU uses a more space-efficient management of
 partial vector accesses, the
 coherence protocol itself involves explcitly intersection tests with
 existing copies. 
 Verifying such framework would require 
 a modification of our abstract state representation and a modification of the access 
 mode translational semantics.
%% If you have bibdatabase file and want bibtex to generate the
%% bibitems, please use
%%
\section*{References}
\bibliographystyle{elsarticle-num} 
\bibliography{bibliography}



\end{document}
\endinput
%%
%% End of file `elsarticle-template-num.tex'.
